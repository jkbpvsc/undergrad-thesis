\section{Authentication}

Authentication the process of proving a claim or an assertion.
Today it is most commonly used in information security, however methods of authentication are not limited to computer science and are also used in fields of archeology, anthropology and others.
\bigskip
\newline
In computer science authentication is used for establishing access between restricted system resources and users through digital identities.
Government and international institutions have developed guidelines for managing digital identities and authentication processes \cite{grassi2017} .
\bigskip
\newline %TODO: Remove
While both humans and other computer systems can be authenticated, we are focusing on how authentication of a human end user.




\subsection{Authentication Process Components}
Authentications is a method used in information security to manage access between restricted system resources and an external user or system wishing to access them.

As defined in RFC-4949 \cite{shirey2007internet}, authentication is \textit{the process of verifying a claim that a system entity or system resource has a certain attribute value.}
This is a broad definition, and it most frequently applies to the verification of users identity (e.g at login), however assertions can be made and verified about any subject or object.
The process of authentication is done in two parts, \textit{identification} and \textit{verification}.

\paragraph{Identification} Presenting an identifier to the authentication system, that establishes the entity being authenticated, this is commonly a username or an email address.
The identifier needs to be unique for the entity it identifies.

The process of identification is not necessarily externally visible, as the identity of the subject/object can be implicit from the user context. 
For example an identifier can be determined by an IP address the user wants to authenticate from, or a system might only have a single identity that can authenticate.

\paragraph{Verification} Presenting or generating authentication information that can be used to verify the claim.
Commonly used authentication information are passwords, one-time tokens, digital signatures.

\subsection{Authentication Factors}

As described in \cite{council2005authentication} authentication systems can rely on three distinct "factors".

\begin{itemize}
	\item \textbf{Knowledge factors} - Something the user \textbf{knows} (e.g, password, security question, PIN)
	\item \textbf{Ownership factors} - Something the user \textbf{owns} (e.g, ID card, security tokens, mobile devices)
	\item \textbf{Inherence factors} - Something the user \textbf{is} or \textbf{does} (e.g, static biometrics - fingerprints, retina, face. dynamic biometrics - voice patterns, typing rhythm)
\end{itemize}

\paragraph{Strong authentication} As defined by governments and financial institutions \cite{schaeffer2010national, ecb2013recommendations}, secure authentication is based on two or more authentication factors. 
This is also referred to as \textit{multi-factor authentication}.

\newpage