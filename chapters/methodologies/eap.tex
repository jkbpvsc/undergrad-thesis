\section{Extensible Authentication Protocol}

Extensible Authentication Protocol \cite{aboba2004extensible} (EAP) is a general purpose authentication framework, designed for network access authentication, where IP might not be available. 
It runs directly over the data link layer such as PPP  \cite{simpson1994rfc1661} and IEEE 802.

EAP defines a set of messages that support negotiation and execution of a variety of authentication protocols.

%EAP - PROTOCOL IN DETAIL

\subsection{Overview}
EAP is a two-party protocol between a \textit{peer} and an \textit{authenticator} at the each end of a link. In the protocol the peer is authenticating with the authenticator.

The protocols is initiated by the authenticator sending a message to the peer, they exchange messages until the authenticator can authenticate the user or not.

\subsection{Messages}

\begin{center}
	\begin{tabular}{|c|c|c|c|c|}
		\hline
		1  & 1 & 2 & 1 & $n \le 2^{16}$\\
		\hline
		Code & Identifier & Length & Type & Type-Data\\
		\hline
	\end{tabular}
\end{center}

%TODO: EAP Packet with table

\subsubsection{Code Field}
The code field determines who the packet is intended for and how or even should the recipient respond.
\bigskip
\newline
%\qquad
\begin{tabular}{rl}
	Code & Name\\
	\hline
	1 & Request\\
	2 & Response\\
	3 & Success\\
	4 & Failure\\
\end{tabular}

\paragraph{Request and Response Packets} 
\textit{Request} packets are sent from by the authenticator to the peer. The peer processes the packet and sends back a \textit{response} packet to the authenticator. 
The response packet needs to have the same identifier as the request packet.

The authenticator will discard response packets that don't have a \textit{matching} identifier with the request packet.
The peer does not re-transmit response packets, but relies on the authenticator to re-transmit a request packet after some time if the matching response is lost.

\paragraph{Success and Failure Packets}

After the authenticator authenticates the peer he sends a \textit{success} packet to the peer. 
If the peer cannot be authenticated, the authenticator will send a \textit{failure} packet.
Both packets signal the end of the authentication process and the peer doesn't need to respond to them.

%EAP - METHODS

\subsubsection{Request Types}
The \textit type field of a packet indicates the format of the type-data field and the methods used to process the data. 
First three types are special purpose types.
\bigskip
\newline
\begin{tabular}{rl}
	Type & Name\\
	\hline
	1 & Identity\\
	2 & Notification\\
	3 & Nak\\
	254 & Expanded Type\\
\end{tabular}
\bigskip
\newline

\paragraph{Identity} Used to query the identity of the peer. The type is often used as an initial message from the authenticator the peer.

\paragraph{Notification} Used to convey an informative message to the peer, by the authenticator. Usage of this type is entirely optional.

\paragraph{Nak} Used only as a response to a request, where the desired authentication type is not available.
The peer includes desired authentication methods, indicated by their type number.
This type is also referred to as Legacy Nak, when compared to \textit{Expanded Nak} (sub-type of the Expanded Type).

\paragraph{Expanded Type} The type field in the EAP packet is 1 octet long, and can represent 256 distinct values.
\textit{Expanded types} expand the space for available method types by adding a \textit{Vendor-ID} field and a \textit{Vendor-Type}.
\bigskip
\begin{center}
	\begin{tabular}{|c|c|c|c|c|c|c|}
		\hline
		1 & 1 & 2 & 1 & 3 & 4 & $n$\\
		\hline
		Code & Identifier & Length & Type & Vendor-ID & Vendor-Type & Data\\
		\hline
	\end{tabular}
\end{center}
\bigskip
When a peer does not support the authentication method requested in an Expanded Type request it needs to respond with an Expanded Nak response. 
If the peer lack support for expanded types, it needs to respond with a legacy nak.

\paragraph{Authentication Methods}
The remaining types correspond to different authentication methods.
According to IANA \cite{joseph2004eap} 49 authentication methods have been assigned type numbers.
The original RFC \cite{aboba2004extensible} already assigned 3 authentication protocols.
\bigskip
\newline
\begin{tabular}{rl}
	Type & Name\\
	\hline
	4 & MD5-Challenge\\
	5 & One-Time Password\\
	6 & Generic Token Card\\
\end{tabular}
\bigskip
\newline
Some notable examples are EAP-TLS \cite{simon2008eap}, EAP-PSK \cite{bersani2007eap}.
EAP SRP-SHA1 \cite{carlson135eap} is especially interesting as the peer uses a ZKP to authenticate itself.

\subsection{Pass-Through Behaviour}
An authenticator can acts as a \textit{Pass-Through Authenticator}, by using the authentication services of a \textit{backend authentication server}.
In this mode of operation the authenticator is relaying the EAP messages between the peer and the backend authentication server.
In IEEE 802.1x the authenticator communicates with a RADIUS server \cite{congdon2003ieee}.

\subsection{IEEE 802.1x}

IEEE 802.1x is a port based network access control standard for LAN and WLAN.
It is part of the IEEE 802.11 group of network protocols.

IEEE 802.1x defines an encapsulation of EAP for use over IEEE 802 as EAPOL or "EAP over LANs".
%TODO: NOT clear if EAP used in WPA2-PSK
EAPOL is used in widely adopted wireless network security standards WPA2. 
In both WPA2-Personal and WPA2-Enterprise, EAPOL is used for communication between the supplicant and the authenticator.

With WPA2-Enterprise, the authenticator functions in a pass-through mode and uses a RADIUS server to authenticate the supplicant.
EAP packets between the authenticator and the authentications server (RADIUS) are encapsulated as RADIUS messages \cite{aboba2003radius, chen2005extensible, congdon2003ieee}