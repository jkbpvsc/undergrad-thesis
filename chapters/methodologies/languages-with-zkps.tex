\section{Languages with Zero-Knowledge Proof Systems}
The zero-knowledge property of interactive proofs is determined by the language the proof exists for.
The choice of language also determines the ZKPs practical applicability.
\bigskip
\newline
The original ZKP protocols \cite{GMR} were proposed for the languages of Quadratic Residuosity problem (QRP) and Quadratic Non-Residuosity Problem (QNRP).
Other simpler protocols are also based on the Discrete Logarithm problem \cite{wu1998secure} and Graph Isomorphism Problem \cite{goldreich2019proofs}.

It has been proven in \cite{GMW} that every language in \textbf{NP} has a ZKP system.

ZKP and interactive protocols have also been used as a tool for studying language complexity \cite{shamir1992ip}.
\bigskip
\newline
In this thesis we are focusing the language QRP.

\subsection{Zero-Knowledge Proof of Quadratic Residuosity Problem}
\textit{Quadratic Residuosity Problem} was used in the original ZKP protocol in the founding paper \cite{GMR}. QRP has a \textit{perfect} zero-knowledge proof system.

QRP is much older than the \cite{GMR} paper, it was first described by Gauss in 1801 \cite{gauss1801disquisitiones}.

\subsubsection{Quadratic Residues} \cite{andrews1994number}
Quadratic residues come from modular arithmetic, a branch of number theory.
\bigskip
\newline
For $a, n \in \mathbb{Z}$, $n > 0$, $\gcd(a, n) = 1 $.
$a$ is a \textit{quadratic residue} if  $\exists x:x^2 \equiv a \Mod{n}$, otherwise $a$ is a \textit{quadratic non-residue}.
\bigskip
\newline
When $n$ is an odd prime, $a$ is a quadratic residue modulo $n$, if and only if.

$$a^{\frac{n-1}{2}} \equiv 1 \Mod{n}$$

\paragraph{Legendre Symbol}
$\dlegendre{a}{p}$ simplifies computations with quadratic residues.
\bigskip
\newline
If $p$ is an odd prime then,

\begin{center}
	$\dlegendre{a}{p} =
		\begin{cases}
			1 & \text{if $a$ is a quadratic residue modulo $p$}\\
			0 & \text{if p $\arrowvert$ a}\\
			-1 & \text{otherwise}
		\end{cases}$
\end{center}

\paragraph{Jacobi Symbol}
A generalised definition of the Legendre symbol $\dlegendre{a}{m}$, to allow the case where $m$ is any odd number.

If $m = p_1p_2 \cdots p_n$, where $p_i$ are odd primes, then
$$\dlegendre{n}{m} = \dlegendre{n}{p_1}\dlegendre{n}{p_2} \cdots\dlegendre{n}{p_n}$$

\subsubsection{Prime Factorization}
\cite{andrews1994number} The \textit{Fundamental Theorem of Arithmetic} states that for each integer \newline $n > 1$, exist primes $p_1 \le p_2 \le \cdots \le p_r$, such that $n = p_1 p_2 \cdots p_r$.
\bigskip
\newline
The process of prime factorization is a decomposition of a number $n$ to its prime factors $p_1 p_2 \cdots p_r$.

Currently no efficient algorithm exists for prime factorization. The problem is especially hard when factoring \textit{semiprimes}, a product of two prime numbers.
This hardness of this problem is used as a core building block in modern asymmetric cryptography like RSA \cite{rivest1978method}.


\subsubsection{Quadratic Residuosity Problem}
Given $a$, semiprime $n = pq$, where $p$ and $q$ are unknown different primes, and Jacobi symbol $\dlegendre{a}{n} = 1$.
\newline
Determine wether $a$ is a quadratic residue modulo $n$ or not.
\bigskip
\newline
The Jacobi Symbol can be efficiently computed using the \textit{Law of Quadratic Reciprocity}, but it does not always tell us if $a$ is quadratic residue modulo $n$ or not. 

$$\dlegendre{a}{n} =\dlegendre{a}{p}\dlegendre{a}{q}$$
If $\legendre{a}{n} = 1$ then $a$ is a quadratic residue both modulo $p$ and $q$, or $a$ is a quadratic non-residue both modulo $p$ and $q$.
To know wether $a$ is a quadratic residue modulo $n$ or not, we would have to know the prime factorization $p, q$ of $n$.
\newline
If $\legendre{a}{n} = -1$ we know $a$ is a quadratic non-residue modulo $p$ or $q$.


\subsubsection{ZKP Protocol for the Quadratic Residuosity Problem}

\begin{center}
	\begin{tabular}{rl}
		$n$ & Semiprime, where $\legendre{x}{n} = 1$\\
 		$x$ & Public input, where $x = w^2 \Mod n$\\
 		$w$ & Provers private input\\
	\end{tabular}
\end{center}


\begin{center}
	\begin{tabular}{r|c|l}
		Prover && Verifier\\
		\hline
		$u \leftarrow \Bbb{Z}_{n}^{*}; y = u^2 \Mod n$ & $\xrightarrow{y}$\\
		& $\xleftarrow{b}$ & $b \leftarrow_R \{0, 1\} $\\
		$z = uw^b\Mod n$ & $\xrightarrow z$ & verify $z^2 = yx^b \Mod n$\\
	\end{tabular}
\end{center}
This protocol is repeated $m$ times, for a probability of error of $\frac{1}{2^m}$.

\subsection{Computational Complexity Classes}

\subsubsection{Non-deterministic Polynomial Time}

\textbf{NP} is a class of problems solvable by a non-deterministic Turing machine in polynomial time. 
Or rather proof of any language in NP can be verified by a deterministic Turing machine in polynomial time.
\bigskip
\newline
Article \cite{GMW} proved that every language in NP has a zero-knowledge proof system, by creating a ZKP protocol for the Graph 3-Colouring problem (3-COL).

\textit{Minimum Colouring Problem} is a problem in graph theory, of what is the minimal $k$ \textit{proper} colouring of a graph, where no adjacent vertices are the same colour.
An instance of ($k=3$) colouring (3-COL) is proven to be \textit{NP-Hard} because a polynomial reduction exists from \textit{Boolean-Satisfiability problem} (3-SAT) to 3-COL \cite{mouatadid2014introduction}.
According to Cook's theorem \cite{cook1971complexity} 3-SAT is \textit{NP-Complete}, and any language in $L \in NP$ can be reduced to and instance of 3-SAT. 
Furthermore because polynomial reductions are \textit{transient}, any language $L \in NP$ can be reduced to an instance of 3-COL.

\subsubsection{Bounded-Error Probabilistic Polynomial Time Languages}

\textbf{BBP} is a class of problems that can be verified by a probabilistic Turing machine in polynomial time.

Trivially every language in BPP has a ZKP system, where the prover sends nothing to the verifier, the verifier checks the proof of $x \in L$ and outputs a the verdict. %TODO: Reword?

\subsection{Alternative Composition of Zero-Knowledge Proofs}
Zero-Knowledge Proofs can alternatively be composed in parallel as compared to sequential composition in \cite{GMR}.
Parallel composition is very interesting practically as it can help reduce the inefficiencies of communication between the prover and the verifier, especially over high latency networks.
\bigskip
\newline
In \cite{goldreich1996composition} they proved that only languages in BPP have 3-round interactive proofs that are zero-knowledge.

The QRP is not believed to be in BPP, so a parallel composition of QRP has weaker notion of zero-knowledge.


