\section{Authentication Protocol using Zero-Knowledge Proofs}

One of the original ZKP protocol proposed in \cite{GMR} was based on the quadratic residuosity problem.

The protocol can be used as a password-based authentication protocol, where the proof proves the possession of a password.
The protocol can further enhanced by similar protocols, to make it meet the security standards of modern password-based authentication systems.

\subsection{Original Protocol} %TODO: Restructure to focus soley on ZKP-QRP as PBA
\bigskip
\begin{center}
	\begin{tabular}{rl}
		$n$ & Semiprime, where $\legendre{x}{n} = 1$\\
 		$x$ & Public input, where $x = w^2 \Mod n$\\
 		$w$ & Password\\
	\end{tabular}
\end{center}
\bigskip
\begin{center}
	\begin{tabular}{rr|c|l}
		& Prover && Verifier\\
		\hline
		1 & $u \leftarrow \Bbb{Z}_{n}^{*}; y = u^2 \Mod n$ & $\xrightarrow{y}$\\
		2 & & $\xleftarrow{b}$ & $b \leftarrow_R \{0, 1\} $\\
		3 & $z = uw^b\Mod n$ & $\xrightarrow z$ & verify $z^2 = yx^b \Mod n$\\
	\end{tabular}
\end{center}
This protocol is repeated $m$ times, for a probability of error of $\frac{1}{2^m}$.
% Security
\subsection{Security}
The protocol is secure against active attacks like masquerading and replay-attacks. 
Zero-knowledge also makes it secure against eavesdropping.

The main issue with the protocols as a password based authentication method is vulnerability to dictionary attacks and attacks pre-computed tables.

%TODO Maybe use passive/active attack terminology
\subsubsection{Password Cracking Vulnerability}

The input $x$ is used by the verifier to verify the witness, it is computed from the private input $w$ as $x = w^2 \Mod n$.
The provers private input $w$ is the password.

The need of the verifier to access the raw value of $x$ prevents the authentication system from processing $x$ with modern password key-derivation methods.
This creates a vulnerability for attacks with pre-computed tables.
An attacker can pre-compute the values of $x$ and compare them with the stored $x$ data by the verifier.

% Key derivation function
% TODO Pick a different title
\subsubsection{Prover Password Key-Derivation}
To utilise PKDF, we need to apply it to derive the provers private input $w$.
Instead of the password being used directly as $w$, the password is processed by a PKDF, and the derivation is used as $w$.

This approach is similar to the one used in \cite{wu1998secure} the Secure Remote Password protocol.
Using a KDF $H$, a random salt $s$ and password $P$, we can derive $w$ and $x$.

$$w = H(P, s)$$
$$x = w^2 \Mod n$$

\subsection{Protocol} %TODO: Pick better title
Using the terminology in NIST Digital Identity Guidelines \cite{grassi2017}. %TODO: Make a mapping between ZKP terminology and NIST DIG
To draw parallels between this terminology and the terminology used in the ZKP-QRP \cite{GMR}. The Prover is the Claimant and Applicant, and the Verifier is the Authenticator ant the CSP.

\paragraph{Values}
\begin{center}
	\begin{tabular}{rl} %TODO: USE DIG Terminology
		$q, p$ & Primes, where $q \ne p$\\
		$n$ & Semiprime modulus, where $n = qp$\\
		$P$ & Credential password \\
		$I$ & Credential identifier \\
		$H$ & PKDF \\
		$s$	& Salt\\
		$w$ & Password hash, where $w = H(P, s)$\\ %TODO check the correct term for this
		$x$ & Integer, where $x = w^2 \Mod{n}$ %TODO check the correct term for this
	\end{tabular}
\end{center}


\paragraph{Enrolment} In the enrolment process the CSP provides the $n$ modulo value to the Applicant.
The Applicant generates a random salt $s$ and computes a private $w$ value from the password $P; w = H(P, s)$.
Applicant next computes $x = w^2 \Mod{n}$ and submits the identifier $I, x, s$ to the CSP.

\bigskip

\begin{center}
	\begin{tabular}{rl|c|l}
		& Applicant & & CSP\\
		\hline
		1 & & $\xleftarrow{n}$ \\
		2 & $s \leftarrow_R \Bbb{Z}$ & $\xrightarrow{I, x, s}$ \\ & $w = H(P, s)$ & \\ & $x = w^2 \Mod{n}$ &
	\end{tabular}
\end{center}

\bigskip

%TODO Make sure this checks out.
CSP binds $x$ and $s$ as the authenticator to the credential $I$.

\paragraph{Authentication}

Authentication happens in two part, in the first part required data is exchanged between the Claimant and the Authenticator. The Claimant identifies himself and the Authenticator provides the semiprime modulus $n$ and the salt $s$.
The second part of the protocol is the ZKP-QRP \cite{GMR} protocol executed between the Claimant and the Authenticator.
\bigskip

\paragraph{First Part (Setup)}

The Claimant sends an identifier $I$ to the Authenticator, which responds with modulo $n$ and the salt $s$. The Claimant uses both values to compute the private input $w$ of the ZKP-QRP protocol.

\bigskip


\begin{center}
	\begin{tabular}{rl|c|l}
  		& Claimant & & Authenticator\\
  		\hline
		1 & & $\xrightarrow{I}$ & \\
		2 & $w = H(P, s)$ & $\xleftarrow{n, s}$ & \\
	\end{tabular}
\end{center}

\paragraph{Second Part (Verification)}
This part is same as the ZKP-QRP protocol described in the \cite{GMR}.
\bigskip
\begin{center}
	\begin{tabular}{rl|c|l}
		& Claimant & & Authenticator \\
		\hline
		1 & $u \leftarrow_R \Bbb{Z}_{n}^{*}$ & $\xrightarrow{y}$ \\
		& $y = u^2$ & \\
		2 & & $\xleftarrow{b}$ & $b \leftarrow_R \{0, 1\} $ \\
		3 & $z = uw^b \Mod n $ & $\xrightarrow{z}$ & verify $z^2 \equiv yx^b \Mod{n}$\\ 
	\end{tabular}
\end{center}
\bigskip
The second part is repeated $m$ times, for a probability of error of $\frac{1}{2^m}$

% \subsubsection{Security}
% Enrolment

% Authentication
%TODO: Add security segment

%Unlike PAP, the pass-
%   word never appears on the wire.  Unlike CHAP (and variants MS-CHAPv1
%   [RFC2433] and MS-CHAPv2 [RFC2759]), access to a cleartext password is
%   not required for the authenticator.  Unlike all of these authentica-
%   tion protocols, SRP is resistant to dictionary attacks against the
%   over-the-wire information.  SRP is also resistant to eavesdropping
%   and active attacks.  As a side-effect, SRP also creates a session key
%   that is resistant to man-in-the-middle attacks and can be used for
%   data encryption.

% TODO: Similar solutions like SRP