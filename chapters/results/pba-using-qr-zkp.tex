\section{System Design}
\label{label:protocol-design}
The main goal of our authentication protocol was to enable password authentication using zero-knowledge proof based on the quadratic residuosity problem. 
The computations used to assert the zero-knowledge proof present a vulnerability when used with passwords.
We extend the protocol with key stretching to protect the low entropy passwords.
The integration of key stretching is not as trivial as it might seem because of the underlying zero-knowledge protocol. 
We can overcome mathematical limitations imposed by the ZKP protocol by separating the data layer where all key stretching operations are done before the ZKP protocol.

In this section we will refer to the \S\ref{zkp-qrp} ZKP protocol of quadratic residuosity as the \textit{original protocol}.

\subsubsection{Vulnerability}
Our use case is for password authentication, which features a unique vulnerability, resulting from properties of passwords themselves, we've explored this topic in \S \ref{label:password-vulnerabilities}.
In particular the original protocol is vulnerable to offline attacks with pre-computed tables.
This vulnerability is caused the operation $x = w^2 \Mod{n}$ used to derive the quadratic residue $x$ mod $n$, which we later prove as a quadratic residue by proving the knowledge of secret $w$.
Intuitively the computation of this equation is relatively inexpensive when compared to special key-stretching function like \textit{Argon2} \cite{biryukov2016argon2}, allowing an attacker to use a pre-computed hash table or a rainbow table.

\subsubsection{Theoretical Constraints}
The solution seems to be a key stretching, as we've described in \S\ref{paragraph:password-hashing}.
Let's have a look at how the verifier verifies the proof.
On the last step the verifier asserts that
$$ z^2 \equiv yx^b \Mod{n}$$.
If we were to protect the control value $x$, by stretching it with a function $H$
$$H(x, s) = x_H$$,
we can then verify the proof with an inverse function $H^{-1}$
$$z^2 = yH^{-1}(x_H, s)^b$$.
This is possible assuming a polynomial algorithm $H^{-1}$ exists, however since key-stretching methods are based on hashing functions which are one-way functions, we know that the probability of a polynomial algorithm $H^{-1}$ to successfully compute a \textit{pseudo-inverse} is negligibly small, for all positive integers $c$ \cite{goldreich2007foundations}
$$\Pr[H(H^{-1}(H(x))) = H(x)] < |x|^{-c}$$.
Even if given unbounded time and resources, the \textit{pseudo-inverse} $x' = H^{-1}(H(x))$ might not be equal to $x' \not = x$.
The set $I_x$ are values that map into $H(x)$, and since $H$ is not injective we know that $|I_x| > 1$.
Meaning that the probability that $x' = x$ is equal to the $\frac{1}{|I_x|}$ 
$$\Pr[H^{-1}(H(x)) = x] = \frac{1}{|I_x|}$$.

\subsubsection{Solution}
Our system is constructed from two phases, the \textit{setup phase} and the \textit{verification phase}.
The purpose of the setup phase is to derive the parameters used in the verification phase of the protocol.
The users password $p$ is stretched to compute the provers private input $w = H(p, s)$,
the control value $x = w^2 \Mod{n}$ is computed.
The protocol is no longer vulnerable to offline attack with a pre-computed table, since to calculate any value $x$ a unique salt $s$ is required.

\begin{center}
	\begin{tabular}{rrl|c|l}
  		& & Prover & & Verifier\\
  		\hline
		Setup Phase & 1 & $w = H(P, s)$ & & \\
		\hline
		& 1 & $u \leftarrow_R \Bbb{Z}_{n}$ &  \\
		Verification & & $y = u^2$ & $\xrightarrow{y}$ \\
		Phase & 2 & & $\xleftarrow{b}$ & $b \leftarrow_R \{0, 1\} $ \\
		& 3 & $z = uw^b \Mod n $ & $\xrightarrow{z}$ & assert $z^2 \equiv yx^b \Mod{n}$\\ 
	\end{tabular}
\end{center}

After the setup phase has been established, the verification phase can start.
After a completion of a single verification phase the verifier can be confident in the proof with the probability of $\frac{1}{2}$.
Additional repetitions of the verification phase improve the confidence in the proof, with $m$ repetitions yielding a confidence of $1 - \frac{1}{2^m}$.
There is no need to repeat the setup phase before each verification phase, since the provers secret $w$ has already beed derived.

%To overcome our limitation, we will use a "layered" approach, where we will apply password-hashing transformations (non-linear functions), before establishing the linear relationship between $w$ and $x$.
%
%The extended protocol has an additional step where the password $p$ is "hashed" with a salt $s$ and password hashing function $H$ to obtain the value $w$.
%After this step we can use the protocol as described in \S\ref{zkp-qrp}.
%$$w = H(p, s)$$
%$$x = w^2 \Mod{n}$$
%
%
%$$x,w \in \mathbb{Z}_n; x = w^2; x_H = H(x);$$
%$$y, u \in  \mathbb{Z}_n; y = u^2$$
%$$(uw)^2 = yx$$
%$$(uw)^2 = yH^{-1}(x_H)$$





%\subsection{Original Protocol} %TODO: Restructure to focus soley on ZKP-QRP as PBA
%\bigskip
%\begin{center}
%	\begin{tabular}{rl}
%		$n$ & Semiprime, where $\legendre{x}{n} = 1$\\
% 		$x$ & Public input, where $x = w^2 \Mod n$\\
% 		$w$ & Password\\
%	\end{tabular}
%\end{center}
%\bigskip
%\begin{center}
%	\begin{tabular}{rr|c|l}
%		& Prover && Verifier\\
%		\hline
%		1 & $u \leftarrow \Bbb{Z}_{n}^{*}; y = u^2 \Mod n$ & $\xrightarrow{y}$\\
%		2 & & $\xleftarrow{b}$ & $b \leftarrow_R \{0, 1\} $\\
%		3 & $z = uw^b\Mod n$ & $\xrightarrow z$ & verify $z^2 = yx^b \Mod n$\\
%	\end{tabular}
%\end{center}
%This protocol is repeated $m$ times, for a probability of error of $\frac{1}{2^m}$.
%% Security
%\subsection{Security}
%The protocol is secure against active attacks like masquerading and replay-attacks. 
%Zero-knowledge also makes it secure against eavesdropping.
%
%The main issue with the protocols as a password based authentication method is vulnerability to dictionary attacks and attacks pre-computed tables.
%
%%TODO Maybe use passive/active attack terminology
%\subsubsection{Password Cracking Vulnerability}
%
%The input $x$ is used by the verifier to verify the witness, it is computed from the private input $w$ as $x = w^2 \Mod n$.
%The provers private input $w$ is the password.
%
%The need of the verifier to access the raw value of $x$ prevents the authentication system from processing $x$ with modern password key-derivation methods.
%This creates a vulnerability for attacks with pre-computed tables.
%An attacker can pre-compute the values of $x$ and compare them with the stored $x$ data by the verifier.
%
%% Key derivation function
%% TODO Pick a different title
%\subsubsection{Prover Password Key-Derivation}
%To utilise PKDF, we need to apply it to derive the provers private input $w$.
%Instead of the password being used directly as $w$, the password is processed by a PKDF, and the derivation is used as $w$.
%
%%This approach is similar to the one used in \cite{wu1998secure} the Secure Remote Password protocol.
%%Using a KDF $H$, a random salt $s$ and password $P$, we can derive $w$ and $x$.
%%
%%$$w = H(P, s)$$
%%$$x = w^2 \Mod n$$
%%
%%\subsection{Protocol} %TODO: Pick better title
%%Using the terminology in NIST Digital Identity Guidelines \cite{grassi2017}. %TODO: Make a mapping between ZKP terminology and NIST DIG
%%To draw parallels between this terminology and the terminology used in the ZKP-QRP \cite{GMR}. The Prover is the Claimant and Applicant, and the Verifier is the Authenticator ant the CSP.
%%
%%\paragraph{Values}
%%\begin{center}
%%	\begin{tabular}{rl} %TODO: USE DIG Terminology
%%		$q, p$ & Primes, where $q \ne p$\\
%%		$n$ & Semiprime modulus, where $n = qp$\\
%%		$P$ & Credential password \\
%%		$I$ & Credential identifier \\
%%		$H$ & PKDF \\
%%		$s$	& Salt\\
%%		$w$ & Password hash, where $w = H(P, s)$\\ %TODO check the correct term for this
%%		$x$ & Integer, where $x = w^2 \Mod{n}$ %TODO check the correct term for this
%%	\end{tabular}
%%\end{center}
%%
%%
%%\paragraph{Enrolment} In the enrolment process the CSP provides the $n$ modulo value to the Applicant.
%%The Applicant generates a random salt $s$ and computes a private $w$ value from the password $P; w = H(P, s)$.
%%Applicant next computes $x = w^2 \Mod{n}$ and submits the identifier $I, x, s$ to the CSP.
%%
%%\bigskip
%
%\begin{center}
%	\begin{tabular}{rl|c|l}
%		& Applicant & & CSP\\
%		\hline
%		1 & & $\xleftarrow{n}$ \\
%		2 & $s \leftarrow_R \Bbb{Z}$ & $\xrightarrow{I, x, s}$ \\ & $w = H(P, s)$ & \\ & $x = w^2 \Mod{n}$ &
%	\end{tabular}
%\end{center}
%
%\bigskip
%
%%TODO Make sure this checks out.
%CSP binds $x$ and $s$ as the authenticator to the credential $I$.
%
%\paragraph{Authentication}
%
%Authentication happens in two part, in the first part required data is exchanged between the Claimant and the Authenticator. The Claimant identifies himself and the Authenticator provides the semiprime modulus $n$ and the salt $s$.
%The second part of the protocol is the ZKP-QRP \cite{GMR} protocol executed between the Claimant and the Authenticator.
%\bigskip
%
%\paragraph{First Part (Setup)}
%
%The Claimant sends an identifier $I$ to the Authenticator, which responds with modulo $n$ and the salt $s$. The Claimant uses both values to compute the private input $w$ of the ZKP-QRP protocol.
%
%\bigskip
%
%
%\begin{center}
%	\begin{tabular}{rl|c|l}
%  		& Claimant & & Authenticator\\
%  		\hline
%		1 & & $\xrightarrow{I}$ & \\
%		2 & $w = H(P, s)$ & $\xleftarrow{n, s}$ & \\
%	\end{tabular}
%\end{center}
%
%\paragraph{Second Part (Verification)}
%This part is same as the ZKP-QRP protocol described in the \cite{GMR}.
%\bigskip
%\begin{center}
%	\begin{tabular}{rl|c|l}
%		& Claimant & & Authenticator \\
%		\hline
%		1 & $u \leftarrow_R \Bbb{Z}_{n}^{*}$ & $\xrightarrow{y}$ \\
%		& $y = u^2$ & \\
%		2 & & $\xleftarrow{b}$ & $b \leftarrow_R \{0, 1\} $ \\
%		3 & $z = uw^b \Mod n $ & $\xrightarrow{z}$ & verify $z^2 \equiv yx^b \Mod{n}$\\ 
%	\end{tabular}
%\end{center}
%\bigskip
%The second part is repeated $m$ times, for a probability of error of $\frac{1}{2^m}$

% \subsubsection{Security}
% Enrolment

% Authentication
%TODO: Add security segment

%Unlike PAP, the pass-
%   word never appears on the wire.  Unlike CHAP (and variants MS-CHAPv1
%   [RFC2433] and MS-CHAPv2 [RFC2759]), access to a cleartext password is
%   not required for the authenticator.  Unlike all of these authentica-
%   tion protocols, SRP is resistant to dictionary attacks against the
%   over-the-wire information.  SRP is also resistant to eavesdropping
%   and active attacks.  As a side-effect, SRP also creates a session key
%   that is resistant to man-in-the-middle attacks and can be used for
%   data encryption.

% TODO: Similar solutions like SRP