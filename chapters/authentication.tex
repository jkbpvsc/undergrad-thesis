\section{Authentication}

As defined by the RFC-4949 \cite{shirey2007internet}, authentication is "The process of verifying a claim that a system entity or system resource has a certain attribute value."
This is a generic definition, and it most frequently applies to the verification of user identity (e.g at login), however assertions can be made and verified about any subject or object.
\\
\\
The authentication process consists of two steps.
\paragraph{Identification} Presenting an identifier to the authentication system, that establishes the entity being authenticated.
In common user authentication systems this is a username or an email verified in the registration process. The identifier needs to be unique for the entity it identifies.

\paragraph{Verification} Presenting or generating authentication information that can be used to verify the claim.
Commonly used authentication information are passwords, one-time tokens, digital signatures.

\subsection{The NITS Model for Digital Identity}

Digital Identity Guidelines \cite{grassi2017} published by the National Institute of Standards and Technology (NIST) describes a simple digital identity model, that provides a generic authentication framework.

The process has distinct steps of \textit{Enrolment and Identity Proofing} and \textit{Authentication}.

\paragraph{Enrolment and Identity Proofing}

The enrolment process describes process where an applicant becomes a subscriber after being successfully proofed by a CSP.
The subscriber is issued a credential and one or more authenticators.

\paragraph{Authentication}

The claimant begins authentication with the verifier by sharing the credential and the authenticators. The verifier validates binding between the credential and authenticators with the CSP.
An authenticated connection is established between the subscriber and the RP after and assertion is provided by the CSP or the verifier to the RP.

%\paragraph{Glossary}
%
%\begin{itemize}
%	\item Credential Service Provider (CSP) - Service responsible for Enrolment and Identity Proofing
%	\item Applicant - Party applying to become a subscriber through the process of Enrolment and Identity Proofing
%	\item 
%\end{itemize}
%
%\paragraph{}
%
%\paragraph{Credential }
%
%\paragraph{Enrolment and Identity Proofing} In this process an \textit{applicant} is proofed by a \textit{Credential Service Provider} (CSP).
%Identity proofing is a process of verifying the association between the digital identity and the real-world identity of an \textit{applicant}. 
%The strength of identity proofing is of 3 different magnitudes. The first level requires no verification, where all information provided by the applicant is self-asserted. The second and third level require that the provided data is verified with external authorisers.
%
%After successful proofing the applicant becomes a \textit{subscriber} and \textit{authenticators} are shared with the \textit{Verifier}.
%
%\paragraph{Authentication}
%In the process of authentication a subj

\subsection{Authentication Factors}

As described in \cite{council2005authentication} authentication systems can rely on three distinct "factors".

\begin{itemize}
	\item \textbf{Knowledge factors} - Something the user \textbf{knows} (e.g, password, security question, PIN)
	\item \textbf{Ownership factors} - Something the user \textbf{owns} (e.g, ID card, security tokens, mobile devices)
	\item \textbf{Inherence factors} - Something the user \textbf{is} or \textbf{does} (e.g, static biometrics - fingerprints, retina, face. dynamic biometrics - voice patterns, typing rhythm)
\end{itemize}

\textit{Strong authentication} as defined by government and financial institutions is, an authentication procedure based two or more authentication factors. Authentication using two or more factors is also referred to as \textit{multi-factor authentication}.

\subsection{Remote User-Authentication}

\section{Authentication based on ZKPs}


%\subsection{Modern password based authentication systems}
%Modern password based authentication systems usually have two functions.
%
%\paragraph{Setup}
%The user first creates a unique identifier in the system and then picks a password. 
%Both values are submitted to the protected system. The identifier is stored in plain text, while the password is "extended" by a key derivation function and stored.
%
%\subparagraph{Key Derivation}
%Key derivation is a process of "extending" a weak key, by using an additional high entropy value called "salt".
%
%\paragraph{Authentication}
%When the user wants to authenticate with the system, he submits the identifier and the password in plain text.
%The system uses the identifier to locate 
