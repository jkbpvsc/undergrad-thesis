%\section{Non-Interactive Zero-Knowledge proofs}
%
%\subsubsection{Introduction}
%In the execution of any \textit{interactive} zero-knowledge proof protocol the most expensive resource is the \textit{"interaction"} itself. Any cryptographic computations can be done relatively quickly compared to how long it takes to exchange a message between the prover and the verifier a few hundred times.
%
%\bigskip
%
%Blum, Feldman and Micali \cite{BMFPS} showed that a non-interactive zero-knowledge protocol is achievable by sharing a common reference string between the prover and the verifier. The original protocol was single-theorem and required a fresh reference string for each proof.
%
%\paragraph{Properties of zero-knowledge proofs}
%
%\begin{enumerate}
%	\item Interaction: The prover and the verifier talk back and forth.
%	\item Hidden Randomisation: 	The verifier tosses coins that are hidden from the prover and thus unpredictable to him.
%	\item Computational Difficulty: The prover embeds in his proofs the computational difficulty of some other problem.
%\end{enumerate}
%
%Non-interactive zero-knowledge proofs are different as interaction between the prover and the verifier is one directional, and the zero-knowledge no longer relies on the secrecy of the randomisation.
%
%The only property lacking in NIZKPs is deniability.
%
%The paper proved the existence of a NIZK for all languages $L \in NP$, by providing providing a protocol for proving the 3COL problem.
%
%The paper \cite{BMFPS} describes a protocol for proving the Graph 3-Colorability problem, to which any lanugange $L \in NP$ can be reduced to according to the Cook-Levin theorem.
%
%\subsection{Applications}
%Modern NIZKPs zk-SNARK (zero-knowledge succinct non-interactive argument of knowledge) and Bulletproofs are used in privacy preserving cryptocurrencies Zcash and Monero.
%
%
%\subsection{Defintion}
%NIZK are defined between a Probabilistic Turing machine A (Prover) and a polynomial deterministic algorithm (Verifier).
%The definition of NIZK still contains completeness, soundness and zero-knowledge. 
%With a difference in that the definition of zero-knowledge is simpler because the of no interaction between  the verifier and the prover, meaning we do not have to worry about a possibility of a cheating verifier.
%
%\subsection{Fiat-Shamir Heuristic}
%Fiat and Shamir \cite{FA}, described a signature scheme 
