\section{Abstract}
In this thesis I will introduce the notion of zero-knowledge proofs, their variants and formal definitions.
In the second part of the thesis I will present a protocol for password based authentication based on the zero-knowledge proof of quadratic residuosity.
In the last part of the thesis I will document the implementation of the protocol as an EAP authentication method.

\newpage

\section{Introduction}
Something about how privacy and security are becoming ever more important as our world is becoming ever more digitalised and connected.

\subsection{Example}

A famous example of a zero-knowledge proof protocol made by \cite{QJM} is The Strange Cave of Ali Baba.

\bigskip

Ali Baba's cave is a cave with a single entrance, that splits into two tunnels that meet in the middle. Where the tunnels meet is a door that can only be opened with a secret passphrase.

\bigskip

Peggy\footnote{Peggy is acronym for \textbf{Prover}} wants to prove to Victor \footnote{Victor is an acronym for \textbf{Verifier}} that she knows the secret passphrase, but she doesn't want to revel the secret nor does she want to reveal her knowledge of the secret to anyone else besides 

\bigskip

Victor turns away from the entrance of the cave, so he cannot Peggy. Peggy enters the cave and goes into one of the tunnels at random. Victor turns around and tells Peggy which tunnel to come out of. Peggy knowing the secret can pass through the door in the middle and emerge from the tunnel requested by Victor.

\bigskip

Given that Peggy didn't know the secret she would still be able to emerge from the tunnel that she initially entered if Victor requested it. Since Victor is choosing the tunnel at random, Peggy has 50\% chance of entering the correct tunnel, and by repeating this process her chances of cheating become vanishingly small ($\frac{1}{2^n}$).

Inversely if Peggy emerges from tunnels Victor requests, he can be convinced that Peggy knows the secret passphrase with a very high probability ($1 - \frac{1}{2^n}$).

\bigskip

Further more any third party observing the interaction cannot be convinced of the validity of the proof because it cannot be assured that the interaction was truly random. For example, Victor could have told Peggy his questions in advance, so Peggy would produce a convincing looking proof.

\newpage


