\section{Abstract}
In this thesis I will introduce the notion of zero-knowledge proofs, their variants and formal definitions.
In the second part of the thesis I will present a protocol for password based authentication based on the zero-knowledge proof of quadratic residuosity.
In the last part of the thesis I will document the implementation of the protocol as an EAP authentication method. %TODO: Rewrite Abstract

\newpage

\section{Introduction}
Something about how privacy and security are becoming ever more important as our world is becoming ever more digitalised and connected. %TODO: Rewrite Abstract

\subsection{Applications}
Most commonly ZKPs were used in authentication and identification systems, as a way to prove knowledge of a secret. 
Recently however there have been a number of new applications in the cryptocurrency and digital identity spaces.

The cryptocurrency Zcash uses a \textit{non-interactive zero-knowledge protocol} zk-SNARK \cite{bowe2018multi} to prove the validity of transactions, without revealing anything about the recipients nor the amount sent.

The cryptocurrency Monero uses a ZKP protocol Bulletproofs \cite{bunz2018bulletproofs}, to achieve anonymous transactions.

\textit{Idemix} \cite{camenisch2002design} an anonymous credential system for interaction between digital identities relies on CL-signatures \cite{camenisch2001efficient} to prove ownership of a credential offline, without the issuing organisation.
Idemix has been implemented in the open-source Hyperledger Indy project.

\subsection{Example} %TODO: Add picture
A famous example of a zero-knowledge proof protocol made by \cite{QJM} is The Strange Cave of Ali Baba.

\bigskip

Ali Baba's cave has a single entrance, that splits into two tunnels that meet in the middle where there is a door that can only be opened with a secret passphrase.

\bigskip

Peggy (or Prover) wants to prove to Victor (or Verifier) that she knows the secret passphrase, but she doesn't want to revel the secret nor does she want to reveal her knowledge of the secret to anyone else besides Victor.

\bigskip

To do this they come up with a scheme.
Victor turns away from the entrance of the cave, so he cannot see Peggy, as she enters the cave and goes into one of the tunnels at random. 
Victor then turns around and tells Peggy which tunnel to come out of.
Peggy knowing the secret can pass through the door in the middle and emerge from the tunnel requested.

\bigskip

If Peggy didn't know the secret she could still convince Victor, by entering the correct tunnel by luck.
But since Victor is choosing the tunnel at random, Peggy's chance of picking the correct tunnel is 50\%. If Victor were to repeat the process $n$ time, her chances of fooling him become arbitrarily small ($2^{-n}$).

With this process Victor can be convinced that Peggy really knows the secret with a very chance ($1 - 2^{-n}$).

\bigskip

Further more any third party observing the interaction cannot be convinced of the validity of the proof because it cannot be assured that the interaction was truly random. 
For example, Victor could have told Peggy his questions in advance, so Peggy would produce a convincing looking proof.