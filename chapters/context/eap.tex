\section{Extensible Authentication Protocol}

Extensible Authentication Protocol \cite{aboba2004extensible} (EAP) is a general purpose authentication framework, designed for network access authentication, where IP might not be available. 
It runs directly over the data link layer such as PPP  \cite{simpson1994rfc1661} and IEEE 802.

EAP defines a set of messages that support negotiation and execution of a variety of authentication protocols.

%EAP - PROTOCOL IN DETAIL

\subsection{Overview}
EAP is a two-party protocol between a \textit{peer} and an \textit{authenticator} at the each end of a link. In the protocol the peer is authenticating with the authenticator.

The protocol is initiated by the authenticator, by sending a \textit{Request} message to the peer, and the peer responds with a \textit{Response} message in a lock-step fashion. 
The success of the authentication is signalled  with a terminal \textit{Success} or \textit{Failure} message.

\subsection{Messages}

%TODO: EAP Packet with table

\paragraph{Request and Response} %TODO: Change Type to lowecase

Request messages are send from the authenticator to the peer.
Request packets have a Type field that indicates what is requested.
The peer processes the packet according to the Type field and send a Response of the same Type.
The Type of the Request determines the data in the packet.

\paragraph{Success and Failure} %TODO: Display field

After a successful completion of an authentication method an authenticator sends a Success packet to peer. A Failure packet is sent if the peer cannot be authenticated with the authenticator.

%EAP - METHODS

\subsection{Request Types}

The Type field of a Request packet indicates what information is being requested. First three types are special purpose types.

\subsubsection{Identity} \textbf{Type 1}. Used to query the identity of the peer.

\subsubsection{Notification} \textbf{Type 2}. Used to convey a message from the authenticator to the peer.

\subsubsection{Nak} \textbf{Type 3}. Used only as a response to a request, where the desired authentication type is not available.
The peer includes desired authentication methods, indicated by their type number.
This type is also referred to as Legacy Nak, when compared to Expanded Nak (sub-type of the Expanded Type).

\subsubsection{Expanded Type} \textbf{Type 254}. The Type field in the EAP packet is 1 octet long, and can represent 256 distinct values.% TODO: Add table representation
Expanded Types expand the space for available method types by adding a \textit{Vendor-ID} field (3 octets) and a \textit{Vendor-Type} (4 octets).

When a peer does not support the authentication method requested in an Expanded Type request it needs to respond with an Expanded Nak response. 
If the peer lack support for Expanded Types, it needs to respond with a Legacy Nak.

\subsubsection{Authentication Methods}
The remaining types correspond to different authentication methods.
According to IANA 49 authentication methods have been assigned Type numbers \cite{joseph2004eap}.

The original RFC \cite{aboba2004extensible} already assigned 3 authentication protocols.

\begin{itemize}
	\item \textbf{Type 4} - MD5-Challenge
	\item \textbf{Type 5} - One-Time Password
	\item \textbf{Type 6} - Generic Token Card
\end{itemize}

Some notable examples are EAP-TLS \cite{simon2008eap}, EAP-PSK \cite{bersani2007eap}, EAP SRP-SHA1 \cite{carlson135eap}. % TODO: Mention that EAP SRP is interesting as it is ZKP
The IANA list is not accounting for any authentication methods supported with the Expanded Type.

%EAP - USAGES

\subsection{Pass-Through Behaviour}
An authenticator can acts as a "Pass-Through Authenticator", relying on authentication services of a \textit{backend authentication server}. 
In this mode of operation the authenticator is relaying the EAP packets between the peer and the backend authentication server.

In IEEE 802.1x the authenticator communicates with a RADIUS server \cite{congdon2003ieee}.

\subsection{IEEE 802.1x}

IEEE 802.1x is standard for port based network access control for LAN and WLAN. It is part of the IEEE 802.11 group of network protocols.

IEEE 802.1x defines an encapsulation of EAP for use over IEEE 802 as EAPOL or "EAP over LANs".
%NOT clear if EAP used in WPA2-PSK
EAPOL is used in widely adopted wireless network security standards WPA2. In both WPA2-Personal and WPA2-Enterprise, EAPOL is used for communication between the supplicant (wireless station) and the authenticator (access point).

With WPA2-Enterprise, the authenticator (Access Point) functions in a pass-through mode as uses a RADIUS server for authentication services. 
EAP packets between the authenticator and the authentications server (RADIUS) are encapsulated as RADIUS messages \cite{aboba2003radius, chen2005extensible, congdon2003ieee}