\section{Zero-Knowledge Proofs}

Traditional theorem proofs are logical arguments that establish truth through inference rules of a deductive system based on axioms and other proven theorems.
Zero-knowledge proofs in contrast are probabilistic in nature and can \textit{"convince"} the verifier of the truth of a theorem with an arbitrarily small probability of error. %TODO: Reword, make it simpler

First defined by Goldwasser, Micali and Rackoff in \cite{GMR} as an interactive two-party protocol between a prover and a verifier.
The protocol uses the the quadratic residuosity problem to embed its proof.

There are three main ingredients that make interactive zero-knowledge proofs work. %TODO: Move this part to the end of the section

\begin{enumerate}
	\item Interaction - The prover and the verifier exchange messages back and forth.
	\item Hidden Randomization - The verifier relies on randomness that is hidden from the prover, and thus unpredictable from him.
	\item Computational Difficulty - The prover embeds his proof in computational difficulty of some other problem.
\end{enumerate}

\subsection{Interactive Proof Systems}
\textbf{Interactive proof systems} are proof systems between a prover and a verifier.
The prover is a computationally unbounded polynomial time Turing machine and the verifier is a probabilistic polynomial time Turing machine.
\textit{Completeness} and \textit{soundness} are enough to define an interactive proof system. 
%The \textbf{IP} complexity class 

\paragraph{Completeness}

Any honest prover can convince the verifier with overwhelming probability.\\
For each $k \in \mathbb{N}$ and sufficiently large $n$;

$$Pr(x \in L; P(x) = y; V(y) = 1) \ge 1 - \frac{1}{n^k}$$

\paragraph{Soundness}

Any verifier following the protocol will reject a cheating prover with overwhelming probability.\\
For each $k \in \mathbb{N}$ and sufficiently large $n$;

$$Pr(x \notin L; P(x) = y; V(y) = 0) \ge 1 - \frac{1}{n^k}$$


\subsubsection{Other Variants of Interactive Proof Systems}

\paragraph{ Interactive Polynomial Time Complexity}
Any problem solvable by an interactive proof systems is in the class of \textbf{IP}.

\paragraph{Arthur-Merlin protocol} Problems in the class AM, the Arthur-Merlin protocol is similar to IP, with the difference in that its a \textit{public-coin protocol}. Meaning that verifiers internal state is visible to the prover, while in IP the state is hidden.
I has been proven that AM is equally powerful as IP and that AM's public internal state gives the prover no advantage.

\paragraph{Multi Prover Interactive Proofs}

%TODO: Add paragraph

\subsection{Knowledge Complexity}

The term \textit{knowledge complexity} quantifies the knowledge that is "extractable" in a proof. It is mainly used for measuring the degree of \textit{zero-knowledge} in different constructions of ZKPs.
\textit{Interactive zero-knowledge proof systems} are interactive proof systems that present only the proof of $x$ membership in language $L$, without revealing any additional knowledge (e.g why is $x \in L$).

In a ZKP system for $L$, a verifier can in polynomial time extract only the proof of membership in $L$ when interacting with a prover. 
The essence of such a system is the idea that the verifiers "view" of an interaction with a prover, can be "simulated" in polynomial time.
Any interactive protocol is \textit{zero-knowledge} if the probability distribution of observed messages is indistinguishable from a distribution that can be simulated on public inputs.

\subsubsection{Indistinguishability of Random Variables} %TODO: Simplify this

Let $U = \{U(x)\}$ and $V = \{V(x)\}$ be two families of random variables, where $x$ is from a language $L$, a particular subset of $\{0, 1\}^*$.

In the framework for distinguishing between random variables, a "judge" is given a sample selected randomly from either $V(x)$ or $U(x)$.
A judge studies the sample and outputs either a $0$ or a $1$, depending on which distribution he thinks the sample came from.

$U(x)$ essentially becomes "replaceable" by $V(x)$, when $x$ increases and any judges prediction becomes uncorrelated with the origin distribution.
By bounding the \textit{size} of the sample and the \textit{time} given to the judge we can obtain different notions of indistinguishability.

\subparagraph{Equality} Given that $U(x)$ and $V(x)$ are equal, they will remain indistinguishable, even if the samples are of arbitrary size and can be studied for an arbitrary amount of time.

\subparagraph{Statistical Indistinguishability} Two random variables are statistically indistinguishable, when given a polynomial sized sample and an arbitrary amount of time, the judges verdict remains meaningless.

\bigskip

Let $L \subset \{0,1\}^*$ be a language, $U(x)$ and $V(x)$ are statistically indistinguishable on $L$ if,

$$\sum_{\alpha \in \{0,1\}^*} |prob(U(x) = \alpha) - prob(V(x) = \alpha) | < |x|^{-c}$$

for $\forall c > 0$, and sufficiently long $x \in L$. 

\subparagraph{Computational Indistinguishability}%TODO: Simplify this

Two random variables are computationally indistinguishable, if judges verdict remains meaningless given a polynomial sized sample and polynomial amount of time.

\bigskip

Let $L \subset \{0,1\}^*$ be a language, poly-bounded families of random variables $U(x)$ and $V(x)$ are computationally indistinguishable on $L$ if for all poly-sized family of circuits $C$, $\forall c > 0$, and a sufficiently long $x \in L$

$$|P(U, C, x) - P(V, C, x)| < |x|^{-c}$$

Any two families that are \textit{computationally indistinguishable} are considered  \textit{indistinguishable} in general.

\subsubsection{Approximability of Random Variables}%TODO: Simplify this

The notion of approximability described the degree to which a random variable $U(x)$ can be "generated" by a probabilistic Turing machine.

\bigskip

A random variable $U(x)$ is \textit{perfectly approximable} if there exists a probabilistic Turing machine $M$, such that for $x \in L$ $M(x)$ is \textit{equal} to $U(x)$.

$U(x)$ is statistically or computationally approximable if $M(x)$ is statistically or computationally indistinguishable from $U(x)$.

\bigskip

Generally speaking when saying a family of random variables $U(x)$ is \textit{approximable} we mean that it is \textit{computationally} approximable.

\subsubsection{Definition of Zero-Knowledge}

The zero-knowledge property is addressing the absence of meaningful information that can be extracted from the protocol by an honest or an cheating verifier.


The verifiers \textit{view} is all data he sees in the interaction with the prover as well data already possessed by the verifier, for example previous interactions with the prover.

\bigskip

An interactive protocol is \textit{perfectly} zero-knowledge if the verifiers view is \textit{perfectly approximable} for all verifiers. Generally we say an interactive protocol is zero-knowledge when its \textit{computationally} zero-knowledge.