\section{PBA Using ZKPQR Implemented as an EAP Method}

\subsection{EAP Packet Format}
An EAP packet is $n$ octets long.


\begin{center}
\begin{tabular}{|c|c|c|c|c|c|}
	\hline
	1 & 1 & 2 & 1 & 1 & $n - 6$\\
	\hline
	Code & Identifier & Length & Type & Sub-Type & Sub-Type Data\\
	\hline 
\end{tabular}
\end{center}

\paragraph{Code}
The code field is one octet

\bigskip

\begin{tabular}{ll}
	1 & Request \\
	2 & Response\\
\end{tabular}

\paragraph{Identifier} The identifier field is one octet, and is being used to match request and response packets.

\paragraph{Lenght} Two octets long, used to indicate the length of the EAP packet.

\paragraph{Type} One octet long.

\bigskip

\begin{tabular}{ll} %TODO: Change the sex joke
	69 & EAP PB-ZKP-QR \\
\end{tabular}

\paragraph{Sub-Type} One octet long

\bigskip 

\begin{tabular}{ll}
	1 & Salt / Modulo\\ %TODO: Better names
	2 & Send Y \\
	3 & Send Z \\
\end{tabular}

\subsection{Salt / Modulus Sub-Type Request Data Format}

EAP Sub-Type 1 request must be sent after obtaining the peers identity. The identity can be acquired with the EAP-Identity (Type 1) packet, or determined somehow otherwise.

The peers identity is used to look up the password salt $s$ and modulus $n$.

\bigskip

\begin{center}
\begin{tabular}{|c|c|c|}
	\hline
	1 & $4 \le n \le 255 $ & $64 \le m$\\
	\hline
	Salt Length & Salt & Modulus\\
	\hline
\end{tabular}
\end{center}

\paragraph{Salt Length}
A single octet for the length of the Salt field in octets. %TODO: Maybe reword copied from EAP-SRP-RFC

\paragraph{Salt}
A random salt value, should be from 4 octets to 255 octets long.
The max length is determined by the max number able to be encoded in the Salt Length field.

\paragraph{Modulus}
Fills the rest of the message to the length specified by the Length field in the EAP header. %TODO: Maybe reword copied from EAP-SRP-RFC
Should be at least 64 octets (512 bits).

This is the $n$ value in the ZKR-QR protocol, a product of two primes $n = qp$.

\subsection{Salt / Modulus Sub-Type Response Data Format}
After the response is sent the first part of the protocol is successfully finished, and can move to the second part.
The second part of the protocol is repeated for multiple times compared to the first part which only happens once. 

A new $y$ value is required at the start of each iteration, for this reason it is always required in the subtype 2 response \ref{subtype-2-response}. The response \textbf{can} contain the value of $y$, required in the second part of the protocol, according to the data format defined in \ref{subtype-2-response}. However the response can also be empty and serve only as an acknowledgment. %TODO: Badly worded

\subsection{Send Y Sub-Type Request Data Format}
The 

\subsection{Send Y Sub-Type Response Data Format}
\label{subtype-2-response}

\subsection{Send Z Sub-Type Data Format}

\subsection{Send Z Sub-Type Data Format}

