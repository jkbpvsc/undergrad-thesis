\section{PBA Using ZKPQR Implemented as an EAP Method} %TODO: Reword every Sub-Type to Subtype
%TODO: Introduction, explain name

\subsection{EAP Packet Format}
An EAP packet is $n$ octets long.


\begin{center}
\begin{tabular}{|c|c|c|c|c|c|}
	\hline
	1 & 1 & 2 & 1 & 1 & $n - 6$\\
	\hline
	Code & Identifier & Length & Type & Sub-Type & Sub-Type Data\\
	\hline 
\end{tabular}
\end{center}

\paragraph{Code}
The code field is one octet

\bigskip

\begin{tabular}{ll}
	1 & Request \\
	2 & Response\\
\end{tabular}

\paragraph{Identifier} The identifier field is one octet, and is being used to match request and response packets.

\paragraph{Length} Two octets long, used to indicate the length of the EAP packet.

\paragraph{Type} One octet long.

\bigskip

\begin{tabular}{ll} %TODO: Change the sex joke
	69 & EAP PB-ZKP-QRP \\
\end{tabular}

\paragraph{Subtype} One octet long

\bigskip 

\begin{tabular}{ll}
	1 & SETUP \\ %TODO: Better names
	2 & ZKP-QR \\
\end{tabular}

\subsubsection{Subtype 1 Request}

EAP Sub-Type 1 request must be sent after obtaining the peers identity. The identity can be acquired with the EAP-Identity (Type 1) packet, or determined somehow otherwise.

The peers identity is used to look up the password salt $s$ and modulus $n$.

\bigskip

\begin{center}
\begin{tabular}{|c|c|c|}
	\hline
	1 & $4 \le n \le 255 $ & $64 \le m$\\
	\hline
	Salt Length & Salt & Modulus\\
	\hline
\end{tabular}
\end{center}

\paragraph{Salt Length}
A single octet for the length of the Salt field in octets. %TODO: Maybe reword copied from EAP-SRP-RFC

\paragraph{Salt}
A random salt value, should be from 4 octets to 255 octets long.
The max length is determined by the max number able to be encoded in the Salt Length field.

\paragraph{Modulus}
Fills the rest of the message to the length specified by the Length field in the EAP header. %TODO: Maybe reword copied from EAP-SRP-RFC
Should be at least 64 octets (512 bits).

This is the $n$ value in the ZKR-QR protocol, a product of two primes $n = qp$.


\subsubsection{Subtype 1 Response}
The request of this subtype serves to complete the setup phase of the protocol, while the response already provides the $y$ value required at the start of each cycle of the second part of the protocol.

\begin{center}
\begin{tabular}{|c|}
	\hline
	$n$ \\
	\hline
	Square $y$\\
	\hline
\end{tabular}
\end{center}

\bigskip

\paragraph{Square $y$} Computed by the peer, as $y = u^2$, where $u \leftarrow_R \Bbb{Z}_{n}^{*}$. Fills the remainder of the message in $n$ octets.

\subsubsection{Subtype 2 Request}

\begin{center}
\begin{tabular}{|c|}
	\hline
	$1$ \\
	\hline
	Random Bit $b$\\
	\hline
\end{tabular}
\end{center}

\paragraph{Random Bit $b$} A single-bit, at the right-most place. The bit value is randomly chosen by the authenticator. 1 octet long.

\subsubsection{Subtype 2 Response}

\begin{center}
\begin{tabular}{|c|c|c|}
	\hline
	$1$ & $n \le 255 $ & $m$\\
	\hline
	Witness Length & Witness $z$ & Square $y$\\ %TODO: Witness, check if this is the correct term?
	\hline
\end{tabular}
\end{center}

\paragraph{Witness Length} A field one octet in length. Determines the length of the Witness field in octets.

\paragraph{Witness} Fields length is limited by the max value of the Witness Length field at 255 octets.
The witness $z$ is computed by the peer, the computation depends on the value of the random bit $b$ in the request.
If $b=0$, then $z = u$, where $u$ was generated for the subtype 1 response.
If $b=1$, then $z = w \cdot u$,  where $u$ was generated for the subtype 1 response, and $w$ is the provers private input.

\paragraph{Square $y$} Field fills up the remainder of the message. 
Square $y$ is the same value as in the subtype 1 response.
It is generated and sent in the $n$-th cycle, to help verify the witness in the $(n+1)$-th cycle.
Same rules apply as when generating the $y$ value if the response to subtype 1 request.

\subsection{Optimisations}
EAP is a lock-step protocol of request response pairs, each packet is first sent by the authenticator as a request, and the peer returns the message as a response. % TODO: Reword this

A naive mapping of ZKP-SQ messages to EAP packets yields 3 new Request/Response pairs. 
We can reduce the amount of new pairs to 2 instead of 3, by interlacing data shared in each pair.

This way we can obtain faster performance by reducing the number of packet needed needed to be exchanged.

\paragraph{Naive Map}

\begin{center}
	\begin{tabular}{c|rcl}
	Pair & Peer  & $\leftrightarrow$ & Authenticator \\
	\hline
	1 & & $\xleftarrow{\text{s, n}}$ &\\
	&& $\xrightarrow{\textvisiblespace}$&\\
	\hline
	2 & & $\xleftarrow{\textvisiblespace}$&\\
	&& $\xrightarrow{y}$&\\
	\hline
	3 & & $\xleftarrow{b}$&\\
	&& $\xrightarrow{z}$&\\
	\hline
	\end{tabular}
\end{center}

\subparagraph{Pair 1} Exchanged once after the authenticator obtaining the peers identity. The authenticator communicates the salt $s$ and modulus $n$ to the peer, in order for the peer to compute the private input $w$. 
Peers response serves as an acknowledgement of a successful setup.

This pair corresponds to the \textit{setup} part of the protocol.

\subparagraph{Pair 2} The authenticator requests the peer to generate the \textit{square} value $y$ and share it in the response.

This pair corresponds to the \textit{interactive zero-knowledge proof} part of the protocol and is repeated for $m$ times. 

\subparagraph{Pair 3} The authenticator requests the peer to compute the \textit{witness} value $z$, according to the procedure determined by the random bit $b$ in the request data.

This pair corresponds to the \textit{interactive zero-knowledge proof} part of the protocol and is repeated for $m$ times.

\subparagraph{Performance}
With this mapping a successful protocol run of $m$ iterations with a confidence of $1 - 2^{-m}$, would require a minimum of $4m + 3$ packet exchanges.

\bigskip

\begin{tabular}{r|l}
	Packets exchanged & Type\\
	\hline
	2 & Pair 1\\
	$2m$ & Pair 2\\
	$2m$ & Pair 3\\
	1 & Type 2 (Success)\\
\end{tabular}

\paragraph{Interlaced Data Mapping}

\begin{center}
	\begin{tabular}{c|rcl}
	Pair & Peer  & $\leftrightarrow$ & Authenticator \\
	\hline
	1 & & $\xleftarrow{\text{s, n}}$ &\\
	&& $\xrightarrow{y_1}$&\\
	\hline
	2 & & $\xleftarrow{b}$&\\
	&& $\xrightarrow{z, y_{n+1}}$&\\
	\hline
	\end{tabular}
\end{center}

\subparagraph{Pair 1} Exchanged once after the authenticator obtaining the peers identity. The authenticator communicates the salt $s$ and modulus $n$ to the peer, in order for the peer to compute the private input $w$. 
Peers computes the square value $y$ and sends it in the response.

The main difference with the naive mapping is that the peer responds prematurely with $y$, instead of in the response to naive pair 2. %TODO: Reword, weird sentence
Trivially we see, that it is possible and valid, because the modulus value $n$ is provided in the pair 1 request.

\subparagraph{Pair 2}
The authenticator already receiving the square value $y$ can send a request with a random bit $b$. The peer responds by computing the \textit{witness} $z$ according to $b$.
The peer also computes the square value $y_{n+1}$, which is used in the next iteration of the protocol.

This is possible because the computation of square value $y$ is only dependent on the modulus $n$, which is established in the request pair 1.

\subparagraph{Performance}
With this mapping a successful protocol run of $m$ rounds with an error rate $2^{-m}$, would require a minimum of $2m + 3$ packet exchanges.

Comparing the performance of both mappings, the interlaced mapping requires half as many exchanges for the same $m$ rounds of protocol.

$$\lim_{1 \rightarrow \infty} \frac{2x + 3}{4x + 3} = \frac{1}{2}$$

\bigskip

\begin{tabular}{r|l}
	Packets exchanged & Type\\
	\hline
	2 & Pair 1\\
	$2m$ & Pair 2\\
	1 & Type 2 (Success)\\
\end{tabular}

\subsection{Security}
EAP PB-ZKP-QRP is resistant to passive attacks, as a ZKP protocol by nature reveals no information, except the validity of the proof to the verifier.
It is also resistant to replay attacks.

The protocol does not enable mutual authentication, nor helps in deriving a session key that can be used for data encryption.

%\subsection{Salt / Modulus Sub-Type Response Data Format}
%After the response is sent the first part of the protocol is successfully finished, and can move to the second part.
%The second part of the protocol is repeated for multiple times compared to the first part which only happens once. 
%
%A new $y$ value is required at the start of each iteration, for this reason it is always required in the subtype 2 response \ref{subtype-2-response}. The response \textbf{can} contain the value of $y$, required in the second part of the protocol, according to the data format defined in \ref{subtype-2-response}. 
%However the response can also be empty and serve only as an acknowledgment. %TODO: Badly worded
%
%\subsection{Send Y Sub-Type Request Data Format}
%This is initial request of an iteration of the second part of the protocol.
%The request is empty
%
%\subsection{Send Y Sub-Type Response Data Format}
%\label{subtype-2-response}
%
%\subsection{Send Z Sub-Type Data Format}
%
%\subsection{Send Z Sub-Type Data Format}




