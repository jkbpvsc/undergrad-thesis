\chapter{Introduction}
\label{chapter:1}

\noindent
Today privacy is a necessary sacrifice we have to make in order to take part in the digital world, imperative to our modern life.
Every day, more digital systems gain access to our personal information. While this practice is often a necessary evil, many companies seek to exploit this position.
Zero-knowledge proofs have the potential to change how our data exists in the digital space. 
Zero-knowledge proof systems are an intriguing cryptographic phenomenon for proving mathematical statements without revealing \textit{why} they are true.

Cryptocurrencies like Zcash \cite{hopwood2016zcash} use zero-knowledge proofs to confirm transactions while keeping the sender and the recipient anonymous and amounts opaque.
The self-sovereign identity space \cite{tobin2016inevitable} uses zero-knowledge proofs as an essential part in a decentralized and privacy-preserving digital identity infrastructure.
Zero-knowledge proofs enable a system to verify the properties \cite{10.1007/978-3-540-89255-7_15} of sensitive data without seeing it and risking misuse. With these tools, we can prove legal age, financial solvency, or our nationality, while revealing no sensitive information.

Advances like this hint of a future where we will look at our current personal data praxis as feudal and undignified.

\bigskip
\noindent
Our focus will be to define a simple use for zero-knowledge proofs.
We will design an authentication system using a zero-knowledge proof as a password verification method, as an authentication method in the extensible authentication protocol (EAP).
When designing the system, we need to protect ourselves from vulnerabilities of passwords.
However, integrating security methods presents a challenge because of the zero-knowledge proof system.

\paragraph{Structure of the thesis.}
This thesis is composed of three chapters.
In \S\ref{chapter:1} we explain the motivation behind the thesis, the focus of the thesis, and its structure.
In \S\ref{chapter:2} we explore the two key topics of authentication and zero-knowledge proofs.
In \S\ref{chapter:3} we present the architecture of our authentication system and the extensible authentication protocol method definition.

\newpage