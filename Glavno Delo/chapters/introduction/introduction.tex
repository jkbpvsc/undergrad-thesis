\section{Motivation}
Today privacy looks like a sacrifice necessary to take part in the digital world, imperative to our modern life. 
Every day, our lives are more intertwined with digital systems with access to our personal information. While this practice is often a necessary evil, many companies seek to exploit this position.
Zero-knowledge proofs have the potential to change how our data exists in the digital space. 
Zero-knowledge proof systems are an intriguing cryptographic phenomenon for proving mathematical statements without revealing \textit{why} they are true.

Cryptocurrencies like Zcash \cite{hopwood2016zcash} use zero-knowledge proofs to confirm transactions while keeping the sender and the recipient anonymous and amounts opaque.
The self-sovereign identity space \cite{tobin2016inevitable} uses zero-knowledge proofs as an essential part in a decentralized and privacy-preserving digital identity infrastructure.
Zero-knowledge proofs enable a system to verify the properties \cite{10.1007/978-3-540-89255-7_15} of sensitive data without seeing it and risking misuse. With these tools, we can prove legal age, financial solvency, or our nationality, while revealing no sensitive information.

Advances like this hint of a future where we will look at our current personal data praxis as feudal and vulgar.

\bigskip
\noindent
Our focus will be to define a simple use for zero-knowledge proofs.
We will design an authentication system using a zero-knowledge proof as a password verification method, as an authentication method in the extensible authentication protocol (EAP).
When designing the system, we need to protect ourselves from vulnerabilities of passwords.
However, integrating security methods presents a challenge because of the zero-knowledge proof system.

\section{Structure of the Thesis}
This thesis is composed of three chapters.
In \S\ref{chapter:1} we explain the motivation behind the thesis, the focus of the thesis, and its structure.
In \S\ref{chapter:2} we explore the two key topics of authentication and zero-knowledge proofs.
In \S\ref{chapter:3} we present the architecture of our authentication system and the extensible authentication protocol method definition.


%
%In computer systems authentication is a process where a system user asserts their identity via an authentication method.
%There are many authentication systems appropriate for different use cases,
%one of the most common systems for end user authentication is a combination of username and password.
%
%
%Our authentication protocol is using zero-knowledge proofs (ZKP) to verify the users password. 
%ZKPs are proofs that prove noting more than that they are true, this allows us to verify the users password without ever revealing or sending the password or password equivalents over the network.
%
%Conceptually passwords are secrets memorised by the user, and it is often the case that weaker passwords are easier to memorise, additionally many users reuse passwords between different systems.
%When designing a password authentication system, we must keep adopt strategies that mitigate the vulnerabilities of passwords.
%
%
%Password authentication is based on a shared secret between the user and the system. Passwords however require special handling because 
%
%
%
%
%
%In computer systems authentication is a process where a system user asserts their identity.
%
%Extensible Authentication Protocol (EAP) is a general purpose authentication protocol framework, designed for network authentication.
%It defines a set of messages and communication patterns to support the negotiation and execution of authentication protocols.
%EAP is authentication method agnostic and is designed to be extended with new authentication methods.
%
%There are many types of authentication methods appropriate for different use cases and contexts, one of the most common authentication methods is a combination of username and password.
%
%In a secure implementation of a password authentication, the system needs protect itself from shortcomings for passwords.



%%Passwords
%One of the most common authentication use cases is end-user authentication with a username and password.
%
%To securely use password authentication, we have to adapt our system to work around short comings of passwords.
%
%%NIST
%Authentication protocols focus on execution in a context where all necessary data exists. 
%NIST Digital Identity Guidelines outline the complete lifecycle of an authentication system.
%
%%ZKP
%Zero-Knowledge Proofs are proofs that reveal only that something is true, without revealing why it is true.
%A simple ZKP protocol is for the problem of quadratic residuosity, which can be used as a basis for a password authentication protocol within the EAP framework.
%
%The protocol has to be extended to overcome the short comings of passwords.
%
%
%
%
%* Authentication
%* Password authentication
%* EAP
%* ZKP







\newpage