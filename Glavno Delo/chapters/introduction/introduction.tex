\chapter{Introduction}
\label{chapter:1}

\noindent
Today privacy is a necessary sacrifice we have to make in order to take part in the digital world, imperative to our modern life.
Every day, more digital systems gain access to our personal information. While this practice is often a necessary evil, many companies seek to exploit this position.
Zero-knowledge proofs (ZKPs) have the potential to change how our data exists in the digital space. 
ZKP systems are an intriguing cryptographic phenomenon for proving mathematical statements without revealing \textit{why} they are true.

Cryptocurrencies like Zcash \cite{hopwood2016zcash} use ZKPs to confirm transactions while keeping the sender and the recipient anonymous and the transaction amount opaque.
The self-sovereign identity space \cite{tobin2016inevitable} uses ZKPs as an essential part in a decentralized and privacy-preserving digital identity infrastructure.
ZKPs enable a system to verify the properties \cite{10.1007/978-3-540-89255-7_15} of sensitive data without seeing it and risking misuse. With these tools, we can prove legal age, financial solvency, or our nationality, while revealing no sensitive information.

Advances like ZKPs hint of a future where we will look at our current personal data practice as feudal and undignified.

\bigskip
\noindent
The focus of this thesis will be to define a simple use for ZKPs.
We will design an authentication system using a ZKP as a password verification method, as an authentication method in the extensible authentication protocol (EAP).
Moreover, in the system design we also have to consider standard methods for protecting against inherent vulnerabilities of password based systems.
\section{Structure of the thesis.}
This thesis is composed of three chapters.
In \S\ref{chapter:2} we explore the two key topics of authentication and ZKPs.
In \S\ref{chapter:3} we present the architecture of our authentication system and the extensible authentication protocol method definition.
Chapter \S\ref{chapter:4} concludes the thesis and gives some possible ways for future work.

\newpage