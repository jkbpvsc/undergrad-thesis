\chapter{Povzetek naloge v slovenskem jeziku}
V zaključni nalogi najprej predstavimo avtentikacijske sisteme, mehanizme avtentikacije z glesli, ter njihove ranljivosti. Spoznamo se tudi z razširljivim atentikacijskim ogrodjem (EAP) v katerega je naš aventikacijski sistem vmeščen, in dokazz ničelnega znanja (ZKP), ki služijo kot naš mehanizem preverjanja veljavnosti gelsa.

V postopku avtentikacije z geslom, se uporabnik najprej identificira z uporabniškim imenom, nato pa to overi z geslom, ki je poznano samo njemu.
Sistem je zaradi svoje preprostosti zelo ranljiv za napade, ki jih na grobo ločimo na napade s povezavo in brez povezave. 
V zasnovo avtentikacijskega sistema bomo vključili tehniko raztegovanja ključev, ki nas ščiti pred napadi brez povezave.

Naš aventikacijski sistem bo vmeščen kot metoda v razširljivo avtentikacijsko ogrodje (EAP).
EAP je splosno namensko avtentikacijsko ogrodnje, zasnovano za mrezno avtentikacijo.
EAP definira skupek sporocil, prek katerih lahko se overjevalnik in vrstnik vskladita in izvršita množico avtentikacijskih protokolov.
EAP je pogosto uporabljen v brezzicnih sistemih avtentikacije.

Kot mehanizem preverjanja gesel uporabljamo dokaze nicelnega znanja (ZKP).
ZKP so način dokazovanja matematicnih trditev, ki lahko dokazejo, da je trditev resnicna brez, da bi razkrili zakaj je trditev resnicna.
Za razliko od izrekov so ZKP verjetnostni, s tem ko prepricajo preverjevalca, da je nekaj res z zanemarljivo moznostjo napake.
ZKP-ji so popularno orodoje v kritpovalutah kot Zcash, Ethereum ter Solana. Uporabljajo se tudi v sistemih digitalne identiete kot Idemix.

%- interaktivni sistemi dokazovanja
Interaktivni dokazni sistemi so teoreticno ogrodje v katerem so definirani interaktivni ZKP. V takem sistemu dokazovalec poskusa prepricati preverjevalca v resnicnost neke trditve.
V takem sistemu velja, da iskreni dokazovalec lahko preprica preverjevalca v resnicnost neke trditve, ter da prevarantski dokazovalec nebo nikoli preprical preverjevalca, ki pravilno sledi protokolu.

%- nicelno znanje
Dokazi nicelnega znanja dokazejo resnicnost trditve brez, da bi razkrivali zakaj je resnicna.
Kljucna ideja za nicelnim znanjem je nerazlocljivost podatkov, ki jih prejme preverjevalec in podatki, ki bi lahko bili ustvarjeni, brez poznavanja skrivne komponente.
%- primerni problemi za dokaze nicelnega znanja
Zmoznost dokazovanja neke trditve z dokazi nicelnega znanja je odvisna od problema za katerega trditev obstaja.
Vrsta problema doloca tudi nacin uporabe ZKP-ja, z preprostimi protokoli lahko dokazemo poznavanje skrivnost, z naprednimi pa tudi, da se je poljubno vezje pravilno izvedno.
Nas mehanizem preverjanja gesla je uporablja preprosti ZKP protokol osnovan na problemu kvadratnih ostankov.
%- problem kvadratnih ostankov
Problem kvadratnih ostankov se pojavi v modularni aritmetiki pri racunanju kvadratnih ostankov, kjer je modulo zmnozek dveh neznanih prastevil. Lastnosti tega problema ga naredijo primernega za funkcijo "zapornih vrat", kjer je operacija lahko v eno stran in zelo tezka v obrato, ce ne poznamo kljuca.
Problem je tezek, ker se v njemu skriva problem razčlenjevanja pra stevil, za katerega vemo, da ucinkoviti algoritmi ne obstajajo.
V ZKP protokolu nam dokazovalec dokaze, da pozna "skrivni" kjuc brez, da bi ga sam razkril.
%- razredi racunske kompleksnosti
% naaah

%- sistemska arhitektura
Oglejmo si arhitekturo avtentikacijskega sistema, v kateri smo zdruzili mehanizem varne aventikacije z gesli in ZKP protokolim za problem kvadratnih ostankov.

V procesu avtenikacije z gesli se uporabnik identificira s tem ko dokaze, da pozna skrivnost, ki je v principu znana samo njemu.
Postopek dokazovanja poznavanja skrivnosti lahko nadomestimo z ZKP protokolom, ce tretirmo geslo kot skrivno komponento, ki jo dokazuje protokol. S tem ko uporabnik skozi ZKP protokol dokaze, da pozna skrivno komponento, dokaze, da pozna svoje geslo.
Gesla nesmemo uporabiti neposredno kot skrivno komponento, ker nam to ustvari ranljivost na napade brez povezave, zato moremo geslo najprej preoblikovati z metodo raztegovanja kljucev.

Postopek avtnetikacije lahko dodano pohitrimo s tem, da zaporedne korake postopka izvajamo vzporedno. S tem skrajsamo cas aventikacije na kostantno dolzino.

%- preverjanje gesla
%- uporaba raztegovanja kljucev
%- postopek preverjanja z zkp
%- premisleki
%
%- definicija eap metode
Nas avtentikacijski protokol bomo ovili v EAP ogrodju.

%- faza nastavitve
%- faza preverjanja
%
%- zakljucek




