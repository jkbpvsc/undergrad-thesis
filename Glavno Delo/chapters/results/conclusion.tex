\chapter{Conclusion and Future Work}
\thispagestyle{fancy}
\label{chapter:4}
The aim of this thesis was to study the use of ZKPs as an authentication mechanism.
In section \S\ref{label:protocol-design}, we have presented the architecture of an authentication system, which uses a ZKP protocol as the password verification method.
We have described how the ZKP protocol can prove the knowledge of the user's password. The architecture also supports key-stretching for protection against password vulnerabilities discussed in \S\ref{label:password-vulnerabilities}.
In section \S\ref{section:eap-84-definition}, we have encapsulated our authentication system within EAP by defining a specification for a new EAP method.
The specification contains definitions for EAP messages and their handling procedures.

We have been successful in our goal of studying and using the ZKP protocol.
However, upon observation, the system performance is not on par with today's industry standards. The iterative nature of the underlying ZKP protocol accumulates communication latencies, slowing down the system.

\paragraph{Future work.}

\begin{itemize}
	\item The EAP method specification presented in this work can be implemented and tested in a real-world environment.
	\item The ZKP protocol used in this work is a first generation protocol. Today there are many newer protocols that have solved many shortcomings of the older generation ZKPs. Using a newer generation ZKP protocol can improve the performance of the authentication system.
	\item The ZKP protocol we've examined is iterative, which can cause worse performance. We've discussed an alternative proof construction in \S\ref{section:pefromance-considerations}, which we aren't pursuing in this thesis because of assumed weaker strength of zero-knowledge. However, in a real-world application, the performance improvements might justify the theoretical shortcomings.
\end{itemize}