\section{Zero-Knowledge Proofs}
In our authentication system, we wish to use ZKPs as a password verification method.

In this section we explore what ZKPs are on a high level, look at a practical analogy of how they work and also how they are used in real life.
Next we look at what are \textit{interactive proof systems}, the parent of ZKP systems. How to quantify the knowledge exchanged in an interactive proof system and finally what makes an interactive proof system zero-knowledge.

\subsection{Basics}
\textit{Zero-Knowledge Proofs} (ZKPs) are a concept in cryptography for proving the validity of mathematical statements.
What makes them particularly interesting is that ZKPs can prove a statement revealing no information about why a statement is true, hence the term \textit{zero-knowledge}.

In mathematics, theorem proofs are logical arguments that establish truth through inference rules of a deductive system based on axioms and other proven theorems.
ZKPs are probabilistic, meaning they \textit{convince} the verifier of the validity with a negligible margin of error.
We use the term convince, because ZKPs are not absolute truth, but the chance of a false statement convincing a verifier is arbitrarily small. 
The difference in definition is subtle, but we will see what that means in practice further on.

ZKPs were first described by Goldwasser, Micali and Rackoff in \cite{goldwasser1989knowledge} in 1985. 
They proposed a proof system as a two-party protocol between a \textit{prover} and a \textit{verifier}.

To help our understanding, we will explore the strange cave of Ali Baba, a famous analogy for a zero-knowledge protocol from a publication called "How to explain zero-knowledge protocols to your children" \cite{10.1007/0-387-34805-0_60}.

\newpage
\subsubsection{The Strange Cave of Ali Baba}

\bigskip

Ali Baba's cave has a single entrance that splits into two tunnels that meet in the middle, where there is a door that only opens with a secret pass phrase.

\begin{figure}[h]
	\centering
	\includegraphics[height=6cm]{images/zkp}
	\caption{The Strange Cave of Ali Baba}
	\label{fig:strange-cave-of-alibaba}
\end{figure}


\bigskip

Peggy (or Prover) wants to prove to Victor (or Verifier) that she knows the secret pass phrase, but she doesn't want to reveal the secret nor does she want to reveal her knowledge of the secret to anyone else besides Victor.

\bigskip

To accomplish this, they come up with a scheme.
Victor stands in front of the cave and faces away from the entrance, to not see Peggy as she enters the cave, and goes into one tunnel at random.
Victor looks at the entrance, so he can see both tunnels, and signals Peggy which tunnel to come out from.
Peggy, knowing the secret, can pass through the door in the middle and emerge from the tunnel requested.


\bigskip

If Peggy did not know the secret, she could fool Victor, only by entering the correct tunnel by chance.
But since Victor is choosing the tunnel at random, Peggy's chance of picking the correct tunnel is $\frac{1}{2}$. If Victor was to repeat the process $n$ time, her chances of Peggy fooling him become arbitrarily small ($\frac{1}{2^n}$).

This way Peggy can convince Victor that she knows the secret with an arbitrarily high probability of ($1 - \frac{1}{2^n}$).

\bigskip

Any third party observing the interaction cannot be convinced of the validity of the proof because they cannot be assured that the interaction was truly random. 
For example, Victor could have told Peggy his questions in advance, so Peggy would produce a convincing looking proof.


\subsubsection{Applications}
Most commonly, ZKPs are used in authentication and identification systems to prove knowledge of a secret. 
Recently, however, there have been several new applications in the cryptocurrency and decentralised identity systems.

The cryptocurrency Zcash uses a \textit{non-interactive zero-knowledge protocol} zk-SNARK \cite{bowe2018multi} to prove the validity of transactions, revealing nothing about the recipients or the amount sent.

\textit{Idemix} \cite{camenisch2002design} an anonymous credential system for interaction between digital identities relies on CL-signatures \cite{camenisch2001efficient} to prove validity of a credential offline, without the issuing organisation.
Idemix has been implemented in the open-source Hyperledger projects.

ZKPs can also prove that values satisfy complex constraints like range proofs \cite{camenisch2008efficient}.

\newpage
\subsection{Interactive Proof Systems}
\label{section:interactive-proof-systems}
An interactive proof system is a proof system where a \textit{prover} attempts to convince a \textit{verifier} that a statement is true.
The prover and the verifier interact with each other by exchanging data until the verifier is convinced or not.

\begin{figure}[h]
	\centering
	\includegraphics[width=8cm]{images/interactive-proof-system}
	\caption{Interactive Proof System}
	\label{fig:interactive-proof-system}
\end{figure}

The prover is a computationally unbounded polynomial time Turing machine and the verifier is a probabilistic polynomial time Turing machine.
An interactive proof system is defined by properties \textit{completeness} and \textit{soundness}.

\paragraph{Notation}
\begin{description}
	\item $\Pr[A]$: probability of event $A$ happening.
	\item $P(x) = y$: prover $P$, outputs a proof $y$ for statement $x$.
	\item $V(y) = 1$: verifier $V$, verifies proof $y$ and outputs $1$ or $0$.
	\item $L$: Language $L$ is a set of \textit{words} or values $x$, with a property $P$, $L = \{x | P(x)\}$. Often used to described a set of values that are a solutions to a mathematical problem. For example, a problem, find $x$ larger than $3$ and lesser than $7$, has the associated language $L_A$ is $L_A = \{x| 3<x<7\} = \{4,5,6\}$. Proving membership of $x$ in $L$ is equivalent to proving $x$ is a solution to a problem $P$ that defines the language $L$.
\end{description}

\paragraph{Completeness}

Any honest prover can convince the verifier with overwhelming probability.\\
For $x \in L$ and each $k \in \mathbb{N}$ and sufficiently large $n$;

$$\Pr[x \in L; P(x) = y; V(y) = 1] \ge 1 - \frac{1}{n^k}$$

\paragraph{Soundness}

Any verifier following the protocol will reject a cheating prover with overwhelming probability.\\
For $x \notin L$ and each $k \in \mathbb{N}$ and sufficiently large $n$;

$$\Pr[x \notin L; P(x) = y; V(y) = 0] \ge 1 - \frac{1}{n^k}$$

\subsection{Zero-Knowledge}

Zero-knowledge proof systems prove the membership of $x$ in language $L$, revealing no additional knowledge (e.g. why is $x \in L$).\\
The essence of zero-knowledge is the idea that the data the verifier has (from current and past interactions with the prover) is indistinguishable from data that can be simulated from public information.
For example, if we return to our analogy in the introduction. 
Victor wants to record what he sees to analyse, or to prove to someone else that Peggy knows the secret.
Victor records which tunnels he calls and from which Peggy emerges, he doesn't record which tunnel Peggy goes into as he is facing away.
Later on Bill and Monica decide to record a similar scheme without knowing the secret.
Bill records himself calling the tunnels and Monica emerging randomly. Sometimes she emerges from the correct one, other times she doesn't. 
Bill later edits the video to only show the times Monica correctly emerged from the tunnel, as if she knew the secret.
Assuming Bill's video editing skills are good, the videos Bill and Victor recorded are indistinguishable, both videos feature someone calling tunnels and a person correctly emerging. 
While Victor's video recorded a genuine proof, there is no information in the video from where we could prove that.
The only one who can be truly convinced is Victor, because he trusts that his own choices of tunnels to call were truly random.

\subsubsection{Indistinguishability}
Indistinguishability describes the (in)ability of distinguishing between two sets of data. The \textit{data} we are comparing is formalised as a random variable.
\bigskip
\newline
Let $U$ and $V$ be two families of random variables and $x \in L$.
We are given a random sample $x$ from either distribution $U$ or $V$, we study the sample to learn which distribution was the origin of $x$.
$U$ and $V$ are said to be \textit{indistinguishable} when our studying of $x$ is no better than guessing randomly.

\subsubsection{Approximability}

The notion of approximability described the degree to which a process $M$ could \textit{generate} a distribution $M(x)$ that is indistinguishable from some distribution $U(x)$.

Formally, a random variable $U(x)$ is \textit{approximable} if there exists a probabilistic Turing machine $M$, such that for $x \in L$, $M(x)$ is \textit{indistinguishable} from $U(x)$.


\subsubsection{Definition of Zero-Knowledge}
Zero-knowledge is a type of interactive proof system in which we can’t learn any meaningful information, besides the validity of the proof.
\bigskip
\newline
An interactive proof system is \textit{zero-knowledge} if $V(x)$ data available to the verifier is \textit{approximable} by $S(x)$ data that can be generated by a \textit{simulator} $S$ from public information.
This also accounts for additional data that might be available to a cheating verifier, for example past interactions with the prover.

\subsubsection{Strengths of Zero-Knowledge}
There are three levels of zero-knowledge, defined by the strength of indistinguishability.
We have defined \textit{indistinguishability} as the ability of a \textit{judge} to distinguish between random variables $V(x)$ and $S(x)$, by attempting to determine the origin of a sample $x$, taken randomly from either distribution.
The strength of indistinguishability is determined by two parameters, the available \textit{time} and the \textit{size of the sample}.\\
\newline
$V(x)$ represents the verifiers view and $S(x)$ the generated data by the simulator $S$. Or if we return to our earlier analogy, $V(x)$ represents Victors interaction with Peggy, and $S(x)$ a fabricated recording of an interaction between Bill and Monica.

\begin{description}
	\item[Perfect Zero-Knowledge]
	$V(x)$ and $S(x)$ are \textbf{equal} when they remain indistinguishable even when given arbitrary time and an unbounded sample size.
	
	\newpage
	\item[Statistical Zero-Knowledge] Two random variables are \textbf{statistically indistinguishable} when they remain indistinguishable given arbitrary time and a polynomial sized sample.
	\bigskip
	\\
	Let $L \subset \{0,1\}^*$ be a language. Two polynomial sized families of random variables $V$ and $S$ are \textit{statistically indistinguishable} when,
		\medskip
	$$\sum_{\alpha \in \{0,1\}^*} |P[V(x) = \alpha] - P[S(x) = \alpha]| < |x|^{-c}$$
	for all constants $c>0$ and all sufficiently long $x \in L$.
	
	\item[Computational Zero-Knowledge] $V(x)$ and $S(x)$ are \textbf{computationally indistinguishable} when they remain indistinguishable given polynomial time and a polynomial sized sample.
	\bigskip
	\\
	Let $L \subset \{0,1\}^*$ be a language. Two polynomial sized families of random variables $V$ and $S$ are \textit{statistically indistinguishable} for all poly-sized families of circuits $C$ when,
	\medskip
	$$|P[V, C, x] - P[S, C, x]| < |x|^{-c}$$
	for all constants $c>0$ and all sufficiently long $x \in L$.

\end{description}


