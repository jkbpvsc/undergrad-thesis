\section{Extensible Authentication Protocol}
\label{section:eap}
We will define our authentication system as a method in the extensible authentication protocol (EAP) framework.

Extensible authentication protocol \cite{aboba2004extensible} (EAP) is a general purpose authentication framework designed for network access authentication. 
It runs directly over the data link layer, such as PPP \cite{simpson1994rfc1661} and IEEE 802 \cite{10.5555/18422.18423}.
EAP defines a set of messages that support negotiation and execution of a variety of authentication protocols.
EAP is a two-party protocol between a \textit{peer} and an \textit{authenticator} at each end of a link.  

\subsection{Messages}
The peer and the authenticator communicate by exchanging \textit{EAP messages} (Figure \ref{fig:eap-messages}).
The protocol starts with the authenticator sending a message to the peer. They keep exchanging messages until the authenticator can either authenticate the peer or not.

Messages are exchanged in a lock-step manner, where an authenticator sends a message and the peer responds to it. 
The authenticator dictates the order of messages, meaning it can send a message at any point of communication, as opposed to the peer, which can only respond to messages from the authenticator.
Any messages from the peer not in a response to the authenticator are discarded.
\bigskip
\begin{figure}[h]
	\centering
	\includegraphics[width=8cm]{images/eap-messages}
	\caption{EAP Peer and Authenticator Communication}
	\label{fig:eap-messages}
\end{figure}

\subsubsection{Message Structure}
Messages are composed of fields (Table \ref{table:eap-message}), each field length is multiple of an octet of bits.
Each field has a special purpose:

\begin{table}
	\centering
	\begin{tabular}{|c|c|c|c|c|c|}
		\hline
		Length (Octets) & 1 & 1 & 2 & 1 & $n \le 2^{16}$\\
		\hline
		Field & Code & Identifier & Length & Type & Type Data\\
		\hline
	\end{tabular}
	\caption{EAP Message Structure}
	\label{table:eap-message}
\end{table}

\paragraph{Code Field}
Determines who the packet is intended for and how or even should the recipient respond.
The following code values are defined:

\begin{quote}
\begin{description}
	\item[Request]\textit{Code 1}. Messages sent by the authenticator to the peer. Response is always expected.
	\item[Response]\textit{Code 2}. Messages sent by the peer to the authenticator as a reply to a \textit{request} message.
	\item[Success]\textit{Code 3}. Sent by the authenticator, when the peer is successfully authenticated. The peer doesn't respond to the message.
	\item[Failure]\textit{Code 4}. Sent by the authenticator, when the peer cannot be authenticated. The peer doesn't respond to the message.
\end{description}
\end{quote}

\paragraph{Identifier Field}
Used to create a session between the peer and the authenticator.
The authenticator uses the field to match request and response messages, each response message needs to have the same identifier as the request.
The authenticator will discard response messages that don't have a matching identifier with the current request.
The peer does not re-transmit a response message, but relies on the authenticator to re-transmit a request message after some time if the matching response is lost.

\paragraph{Length Field}
Determines the total size of the EAP message. Because EAP provides support for generic authentication methods, the final length of the messages is variable.
The length of the \textit{type data} field entirely depends on the authentication method used.

\paragraph{\textit{Type} and \textit{Type Data} Field}
\label{text:eap:type}
Determines how the message should be processed and how to interpret the \textit{type data} field.
Most message types represent authentication methods, except for four special purpose types.
The \textit{type} used is determined by the authenticator when sending the request message. The response message from a peer needs to be of the same type as the request, except where that type is not supported by the peer.
The following types are defined:

\begin{quote}
\begin{description}
	\label{def:eap-identitiy}
	\item[Identity] \textit{Type 1}. Used to query the identity of the peer. The type is often used as an initial message from the authenticator the peer, however its use is entirely optional and EAP methods should rely on method-specific identity queries.
	
	\item[Notification]\textit{Type 2}. Used to convey an informative message to the peer, by the authenticator. Usage of this type is entirely optional.
	\item[Nak]\textit{Type 3}. Used only as a response to a request, where the desired type is not available.
	The peer includes desired authentication methods, indicated by their type number.
	This type is also referred to as Legacy Nak, when compared to \textit{Expanded Nak} (sub-type of the Expanded Type).
	\item[Expanded Type] \textit{Type 254}. 
	Used to expand the space of possible message types beyond the original 256 possible types.
	The expanded type \textit{data field} is composed from a \textit{Vendor-ID} field, \textit{Vendor-Type} and the type data.
	\bigskip
	\begin{center}
		\begin{tabular}{|c|c|c|c|c|c|}
		\cline{2-5}
		\hline
		Length (octets) & & 3 & 4 & n\\
		\hline
		Field Type & ... & Vendor-ID & Vendor-Type & Vendor-Type Data\\
		\hline
		\end{tabular}
	\end{center}
	\bigskip
	A peer can respond to an unsupported request type with an \textit{expanded nak}, if he desires to use an EAP method supported with the expanded type.
	\item[Experimental] \textit{Type 255}. This type is used for experimenting with new EAP Types and has not fixed format.
\end{description}
\end{quote}

\paragraph{Authentication Methods}
The remaining types correspond to different authentication methods.
IANA \cite{joseph2004eap} assigns type numbers to 49 different authentication methods.
Authors of the original RFC \cite{aboba2004extensible} already assigned 3 authentication protocols:

\begin{quote}
\begin{description}
	\item[MD5-Challenge] \textit{Type 4}. An EAP implementation of the PPP-CHAP protocol \cite{simon2008eap}.
	\item[One-Time Password] \textit{Type 5}. An EAP implementation of the one-time password system \cite{haller1998one}.
	\item[Generic Token Card] \textit{Type 6.} This type facilitates various challenge/response \textit{token card} implementations.
\end{description}
\end{quote}

Some other notable examples are EAP-TLS \cite{simon2008eap}, EAP-PSK \cite{bersani2007eap}.
EAP SRP-SHA1 \cite{ietf-pppext-eap-srp-03} is especially interesting as it uses a ZKP system to verify the peer's secret, similar to our own EAP method.

\subsection{Pass-Through Behaviour}
An authenticator can act as a \textit{Pass-Through Authenticator}, by using the authentication services of a \textit{backend authentication server}.
In this mode of operation, the authenticator is relaying the EAP messages between the peer and the backend authentication server.
For example, in IEEE 802.1x the authenticator communicates with a RADIUS server \cite{congdon2003ieee}.

\paragraph{IEEE 802.1x}

Is a port based network access control standard for LAN and WLAN.
It is part of the IEEE 802.11 group of network protocols.
IEEE 802.1x defines an encapsulation of EAP for use over IEEE 802 as EAPOL or "EAP over LANs".
EAPOL is used in widely adopted wireless network security standards WPA2. 
In WPA2-Enterprise, EAPOL is used for communication between the supplicant and the authenticator.

With WPA2-Enterprise, the authenticator functions in a pass-through mode and uses a RADIUS server to authenticate the supplicant.
EAP packets between the authenticator and the authentications server (RADIUS) are encapsulated as RADIUS messages \cite{aboba2003radius, chen2005extensible, congdon2003ieee}.

\subsection{MD5-Challenge EAP Method}
Let us look at an example of an EAP authentication process and examine the messages exchanged between the peer and the authenticator. This EAP instance uses the MD5-Challenge authentication method.
MD5-Challenge is an EAP method analogous to PPP CHAP \cite{simpson1996ppp}. The message Type 4 denotes the method.

\paragraph{PPP CHAP.}
The PPP \textit{challenge handshake authentication protocol} is an authentication model based on a shared secret between the peer and the authenticator.
The authenticator authenticates the user by first sending him a random challenge $c$, which the user concatenates with the secret $s$ and hashes with a hashing function $d = h(c | s)$.
The hash digest $d$ is returned in the response and the authenticator compares the received hash with the locally computed hash, if they match the peer is authenticated.

Aside from a slightly different message format, the MD5-Challenge authentication process is functionally the same as PPP CHAP, with the difference that while PPP CHAP is hashing algorithm agnostic, MD5-Challenge specifics the use of the MD5 hashing algorithm \cite{rivest1992md5}.

\bigskip
\noindent
Let us examine the individual steps in EAP authentication with this method. 
The steps are visualised in the sequence diagram \ref{fig:eap-md5}.
At each step we will describe what is happening and note the contents of the EAP messages being exchanged, we are omitting the \textit{identifier}, \textit{length} and \textit{type-data} fields of the messages, as their contents are dynamically determined when the protocol is running.

\begin{figure}[h]
	\centering
	\includegraphics[width=10cm]{images/eap-md5-2}
	\caption{EAP (MD5-Challenge)}
	\label{fig:eap-md5}
\end{figure}

\begin{description}
	\item [MD5-Challenge] Message is sent to the peer with a random challenge $c$. $(Code=1, Type=4)$.
	\item [Nak] In the case that MD5-Challenge method is unacceptable to the peer, he should respond with a \textit{Nak} message $(Code=2, Type=3)$, the \textit{type data} of the message can contain the number indicating the preferred EAP method.
	\item [MD5-Response] The peer computes the hash digest $d$ as described before with the MD5 hashing algorithm and sends it in a response. $(Code=2, Type=4)$.
	\item [Success] The authenticator sends this message if the digest $d$ was successfully verified. The peer was successfully authenticated, and the protocol stops. $(Code=3)$.
	\item [Failure] If the digest $d$ isn't valid, the authenticator sends the \textit{failure} message indicating that the peer could not be authenticated. $(Code=4)$.
\end{description}



































