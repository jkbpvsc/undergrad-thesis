\documentclass[12pt]{article}

\title{Undergrad thesis summary}
\author{Jakob Povšič}
\date{2021}

\usepackage[english]{babel}
\usepackage{fontspec}

\begin{document}
	\maketitle
	
%	\section{Uvod}
%	Internet je v zadnjih desetletjih korenito preobrazil našo družbo, in z vsakim dnem je naše življenje bolj zlito z digitalnim.
%	Vendar niso vse spremembe pozitivne, še posebej na področju zasebnosti se danes zdita pojma digitalnega življenja in zasebnosti samo izključujoča.
%	Zaradi poslovnih modelov velikih tehnoloških podjetji, ki temeljijo na dostopanju osebnih podatkov uporabnikov, dobiva zasebnost vsak dan večji pomen.
%	 
%	Dokazi o ničelnem znanju (ZKP) so eno od orodji, ki bi lahko spremenili naš pristop do obdelave podatkov.
%	Njihove aplikacije omogočijo odločitve nad podatki, brez razkrivanja podatkov samih.
%	
%	V zaključni nalogi bomo raziskali uporabo ZKP v avtentikacijskem sistemu, kot metodo dokazovanja pravilnosti gesla.	

	\section{Introduction}
	The Internet has radically changed our society in the last decades, and every day our life is more digital.
	However not all changes are positive, today privacy seems like a necessary sacrifice in our digital lives, and it’s becoming a more important issue because of the business models of big-tech companies that are based on accessing our data.

	
	Zero-knowledge proofs (ZKPs) are one of the tools that could change our approach to working personal data.
	Their applications enable making decisions over data without disclosing the data itself.
	
	In the thesis, we will explore the use of ZKPs in an authentication system, as a method of proving password correctness.
	
	\newpage
	
	%	\section{Avtentikacija in razširljiv avtentikacijsi protokol}
%	Avtentikacija je proces preverjanja resničnosti trditev, ki jih predstavlja neka entiteta o sebi ali predmetih.
%	V informacijski varnosti se avtentikacija pogosto uporablja za vspostavljanje dostopa med uporabniki in zasčitenimi sistemi.
%	Avtentikacija z uporabniškim imenom in geslom je pogost model s katerim se danes srečujejo uporabniki.
%	
%	V zaključni nalogi bomo zasnovali avtentikacijski sistem kot metodo v razširljivem avtentikacijskem protokolu (EAP).
%	EAP \cite{aboba2004extensible} je razširljiv protokol za pogajanje in izvršitev mnogih avtentikacijskih metod (EAP metod).
%	Sistem bo omogočal avtentikacijo z uporabniškim imenom in geslom preko omrežja ter uporabljal ZKP kot mehanizem preverjanja gesla.
%
%	Uporaba gesel prinese določene ranljivosti, zato je industrija posvojila varnostne metode, ki onemogočajo določene napade na sistem.
%	Ena od teh metod je raztegovanje ključev, uporaba ZKP mora omogočati uporabo takšnih metod za zagotavljanje minimalne varnosti.

	
	\section{Authentication and the Extensible Authentication Protocol}
	Authentication is the process of verifying a claim an entity is making about itself or a subject.
	In information security authentication is commonly used for establishing access between users and protected system resources.
	Authentication with a username and password is a common model that everyone encounters daily.
	
	In the thesis, we will design an authentication system as a method in the \textit{Extensible Authentication Protocol} (EAP).
	EAP \cite{aboba2004extensible} is an extensible protocol for negotiation and execution of a variety of authentication methods (EAP methods).
	Our system will support authentication with a username and password over the network and will use a ZKP protocol as a mechanism for checking the password.
	
	Password authentication has vulnerabilities, for which the industry has adopted security methods that prevent their exploitation.
	One of these methods is key-stretching, our authentication system has to use these tools to ensure a sufficient level of security.
	
%	\section{Dokazi o ničelnem znanju}
%	Dokazi o ničelnem znanju \cite{goldwasser1989knowledge} so zanimiv koncept s področja kriptografije, ki omogočajo dokazovanje matematičnih izjav, brez razkrivanja zakaj so resnične.
%	To omogoča izredno zanimive aplikacije, saj lahko takšen sistem izvaja operacije nad podatki, brez da bi do njih dostopal v surovi obliki.
%	V takem sistemu lahko dokažemo, da smo polnoletni oz. državljani neke države 	oz., da imamo določen znesek na bančnem računu, brez razkrivanja občutljivih osebnih podatkov.
%	Kriptovalute kot Monero in Zcash uporabljajo ZKP-je za dokazovanje veljavnosti transakcij, brez razkrivanja identitete pošiljatelja, prejemnika ali vrednosti poslanega zneska.
%	Projekti kot je Idemix \cite{camenisch2002design}, uporabljajo ZKP-je kot orodje za operiranje nad osebnimi podatki na dokumentih, izdanih s strani organizaciji oz. avtoritet.
%		
%	ZKP sistemi so osnovani na vrsti matematičnih problemov.
%	V našem avtentiakcijskem sistemu bomo uporabili ZKP sistem, osnovan na problemu kvadratnih ostankov.
	
	\section{Zero-Knowledge Proofs}
	Zero-Knowledge Proofs \cite{goldwasser1989knowledge} are an interesting concept from the field of cryptography, that enable proofs of mathematical statements without revealing why they are true.
	This enables very interesting applications, where a system can operate with data without accessing them in their raw form.
	In such a system we can prove that we are of legal age or possess valid documents or that we have a sufficient bank account balance without revealing any sensitive information.
	Cryptocurrencies like Monero and Zcash are using ZKPs to validate transactions without revealing the identity of the sender, recipient, or the sum sent.
	Projects like Idemix \cite{camenisch2002design}, use ZKPs to achieve privacy-preserving verifiable credentials.
	
	ZKP protocols are based on a certain class of mathematical problems.
	In our authentication system, we will use the ZKP protocol based on the quadratic residuosity problem.
	
%	\section{Načrtovanje avtentikacijskega sistema kot EAP metodo}
%	Naš avtentikacijski sistem razširja ZKP, sistem osnovan na problemu kvardatnih ostankov, z metodami razširjevanja ključev, kar onemogoča napade nad podatki, ki ji uporablja sam sistem za preverjanje gesla.
%	Avtentikacijski sistem je definiran kot EAP metoda.

	\section{Design of the authentication system as an EAP method}
	Our authentication system extends the ZKP protocol based on the quadratic residuosity problem, with key-stretching methods, which prevents offline attacks over data used for password verification.
	The authentication system is defined as an EAP method.
	
	\bibliographystyle{plain}
	\bibliography{bib}
\end{document}